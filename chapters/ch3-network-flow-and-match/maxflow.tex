\section{最大流}

\subsection{Dinic 算法 (带当前弧优化)}

\par 不断在残量图上使用 BFS 求出结点的层次,构建分层图(即给结点标注 d[i]);然后在分层图上 DFS 寻找增广路,回溯时实时更新剩余容量。每个点可以流向多条出边。时间复杂度为 $O(n^2m)$。
\begin{minted}{c++}
template <int N> struct Dinic {
    const int INF = 1e9;
    struct E {
        int to, cap, rev;// x->y 目标点 to=y 流量cap 反向边G[to][rev]=x
    };
    vector<E> G[N];
    int lev[N], cur[N];// 层数 当前弧优化
    inline void add(int x, int y, int c) {
        G[x].push_back({ y, c, (int)G[y].size() });
        G[y].push_back({ x, 0, (int)G[x].size() - 1 });
    }
    void bfs(int s) {
        queue<int> q;
        memset(lev, -1, sizeof lev);

        for (lev[s] = 0, q.push(s); q.size();) {
            int x = q.front();
            q.pop();

            for (auto &e : G[x])
                if (e.cap > 0 && lev[e.to] < 0)// 层数未赋值 容量大于0
                    lev[e.to] = lev[x] + 1, q.push(e.to);
        }
    }
    int dfs(int x, int t, int f) {
        if (x == t)
            return f;
        // 当前弧优化
        for (int &i = cur[x], sz = G[x].size(), d; i < sz; i++) {
            auto &e = G[x][i];

            if (e.cap > 0 && lev[x] < lev[e.to]) {// lev[e.to]==lev[x]+1
                if ((d = dfs(e.to, t, min(f, e.cap))) > 0) {
                    e.cap -= d, G[e.to][e.rev].cap += d;// 正向边流量-d 反向边容量+d
                    return d;
                }
            }
        }

        return 0;
    }
    int64_t maxflow(int s, int t) {
        for (int64_t flow = 0, f;;) {
            bfs(s);

            if (lev[t] < 0)// BFS未遍历到终点
                return flow;

            memset(cur, 0, sizeof cur);// 当前弧指向第一条边即G[x][0]

            while ((f = dfs(s, t, INF)) > 0)
                flow += f;
        }
    }
};
Dinic<10010> din;
\end{minted}

\subsection{无源汇上下界可行流}
\par \textbf{问题:} 给定一个网络,求一个流满足:每条边的流量处在给定的下界和上界 [lower,upper] 之间,满足流量守恒

\begin{itemize}
\item 记$A(u)=\sum_{to [i]=u}f(i)-\sum_{from[i]=u}f(i)$即流入减去流出
\item 若$A(u)>0$,源点 $S$ 向 $u$ 点连边,容量是$A(u)$
\item 若$A(u)<0$,$u$ 点向汇点 $T$ 连边,容量是$-A(u)$
\end{itemize}
如果\textbf{满流}(虚拟源点 S 流跑满)此时即可求出流2(否则无解),再加上流1即是满足题意的可行流
\begin{minted}{c++}
int main() {

    cin >> n >> m;
    S = 0, T = n + 1; // 虚拟源点 汇点
    for (int i = 1; i <= m; i++) {
        int x, y, c, d; cin >> x >> y >> c >> d;
        din.add(x, y, d - c, i);
        in[y] += c, out[x] += c;
        l[i] = c;
    }
    int sum = 0;
    for (int i = 1; i <= n; i++) {
        if (in[i] > out[i])
            din.add(S, i, in[i] - out[i], 0), sum += in[i] - out[i]; // 满流
        else
            din.add(i, T, out[i] - in[i], 0);
    }
    if (sum != din.maxflow(S, T))
        cout << "NO\n";
    else {
        cout << "YES\n";
        for (int x = 1; x <= n; x++)
            for (auto [y, cap, rev, id] : din.G[x]) {
                ans[id] = cap;
            }
        for (int i = 1; i <= m; i++)
            cout << ans[i] + l[i] << '\n';
    }
    return 0;
}
\end{minted}
\subsection{有源汇上下界最大流}
\par \noindent 连接一条 $t\to s$下界是 0 上界是 $\infty$ 的边,由此转化循环流问题(无源汇上下界可行流)
\par \noindent 
\par \noindent 按照循环流问题建图,首先跑 $S \to T$ 的dinic 是否能够找到一条可行流(即判断是否是满流)然后在换源点和汇点并删去 $t\to s$ 的边再跑一边 $s\to t$ dinic(榨干残留网络),可行流 + 第二次dinic即是最大流
\begin{minted}{c++}
int main() {

    // ............. 无源汇上下界可行流建图
    din.add(t, s, 1e9);

    res = din.maxflow(S, T);

    if (res != sum) {
        cout << "No Solution\n";
        return 0;
    }
    auto &[v, cap, inv] = din.G[s].back();
    res = cap;
    din.erase(din.G[v][inv]); //删去 t->s 容量是1e9这条边
    cout << res + din.maxflow(s, t);
    return 0;
}
\end{minted}

\subsection{有源汇上下界最小流}
\par \noindent 最小流 = 可行流 + $s\to t$ 流,由于可行流固定,为了使结果最小即 $s\to t$ 流最小,由于 $s\to t$ 的流 = $- t\to s$的流,如果 $t\to s$ 求最大流,那么 $s\to t$ 就是最小流,$f_{s\to t}=-f_{t\to s}$ 这也是为什么相减的原因。
\begin{minted}{c++}
int main() {
    
    // ............. 无源汇上下界可行流建图
    din.add(t, s, 1e9);

    res = din.maxflow(S, T);

    if (res != sum) {
        cout << "No Solution\n";
        return 0;
    }

    auto &[v, cap, inv] = din.G[s].back();
    res = cap;
    din.erase(din.G[v][inv]); //删去 t->s 容量是1e9这条边

    cout << res - din.maxflow(t, s); // f(s->t) = - f(t->s)
    return 0;
}
\end{minted}
