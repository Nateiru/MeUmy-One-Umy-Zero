\section{pb-ds 平衡树}

\begin{minted}{c++}
#include<ext/pb_ds/assoc_container.hpp>
#include<ext/pb_ds/tree_policy.hpp>   // 用tree
#include<ext/pb_ds/hash_policy.hpp>   // 用hash
#include<ext/pb_ds/trie_policy.hpp>   // 用trie
#include<ext/pb_ds/priority_queue.hpp>// 用priority_queue
using namespace __gnu_pbds;
---
#include<bits/extc++.h>
using namespace __gnu_pbds;
//bits/extc++.h与bits/stdc++.h类似,bits/extc++.h是所有拓展库,bits/stdc++.h是所有标准库
//=======================================================
#include<ext/pb_ds/assoc_container.hpp>
#include<ext/pb_ds/hash_policy.hpp>

using namespace __gnu_pbds;
template<typename T>
using ordered_set = tree<T,null_type,less<T>,rb_tree_tag,tree_order_statistics_node_update>;

// rb_tree_tag 和 splay_tree_tag 选择树的类型(红黑树和伸展树)
T // 自定义数据类型
null_type//无映射(老版本g++为null_mapped_type)
less<T>//Node的排序方式从小到大排序
tree_order_statistics_node_update//参数表示如何更新保存节点信息 tree_order_statistics_node_update会额外获得order_of_key()和find_by_order()两个功能。

ordered_set<Node> Tree;  // Node 自定义struct 注意重载less
Tree.insert(Node);       // 插入
Tree.erase(Node);        // 删除
Tree.order_of_key(Node); // 求Node的排名:当前数小的数的个数 +1
Tree.find_by_order(k);   // 返回排名为k+1的iterator 即有k个Node比*it小
Tree.join(b);            // 将b并入Tree,前提是两棵树类型一致并且二没有重复元素
Tree.split(v, b);        // 分裂,key小于等于v的元素属于Tree,其余属于b
Tree.lower_bound(Node);  // 返回第一个大于等于x的元素的迭代器
Tree.upper_bound(Node);  // 返回第一个大于x的元素的迭代器

//以上的所有操作的时间复杂度均为O(logn)
//注意,插入的元素会去重,如set
ordered_set<T>::point_iterator it=Tree.begin();  // 迭代器
//显然迭代器可以++,--运算
\end{minted}
\par \noindent  维护 $n$ 棵平衡树,启发式合并暴力合并。平衡树支持第 $k$ 小
\begin{minted}{c++}
#include <bits/stdc++.h>
#include <ext/pb_ds/assoc_container.hpp>
#include <ext/pb_ds/hash_policy.hpp>
using namespace std;
using namespace __gnu_pbds;
template<typename T>
using ordered_set = tree<T, null_type, less<T>, rb_tree_tag, tree_order_statistics_node_update>;
const int N = 500010;
int fa[N];
int n, m;
int find(int x) {
    return x == fa[x] ? x : fa[x] = find(fa[x]);
}
ordered_set<pair<int, int>> rt[N];
void merge(int u, int v) { // u < - v
    if (rt[u].size() < rt[v].size())
        swap(u, v);
    ordered_set<pair<int, int>>::point_iterator it = rt[v].begin();
    for (; it != rt[v].end(); it++) rt[u].insert(*it);
    rt[v].clear();
    fa[v] = u;
}
int main() {
    n = rd(), m = rd();
    for (int i = 1; i <= n; i++) rt[i].insert(pair<int, int>(rd(), i));
    for (int i = 1; i <= n; i++) fa[i] = i;

    while (m--) {
        int u = rd(), v = rd();
        u = find(u), v = find(v);
        if (u != v) merge(u, v);
    }
    int qc = rd();
    while (qc--) {
        char op[4];
        scanf("%s", op);
        int x = rd(), y = rd();
        if (op[0] == 'Q') {
            int u = find(x);
            if (rt[u].size() < y) puts("-1");
            else
                printf("%d\n", rt[u].find_by_order(y - 1)->second);
        } else {
            int u = find(x), v = find(y);
            if (u != v) merge(u, v);
        }
    }

    return 0;
}
\end{minted}