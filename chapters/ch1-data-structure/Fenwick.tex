\section{树状数组}

\par \noindent Tips: lowbit(x) => (x \& (-x)) 的含义:当 x 为 0 时,结果为 0。当 x 为奇数时,结果为 1。当 *x* 为偶数时,结果为 x 中 2 的最大次方的因子。
\subsection{区间修改,区间求和}

\par \noindent 引入差分序列,使树状数组支持区间修改。树状数组中维护的是差分序列,需要有一个数组维护原数据。
\begin{minted}{c++}
namespace Fenwick {
    int c0[maxn], c1[maxn], n;
    
    inline int lowbit(int x) { return x & (-x); }
    
    void add(int k, int v) {
      int i = k * v;
      while (k <= n) {
        c0[k] += v, c1[k] += i;
        k += lowbit(k);
      }
    }
    int sum(int k) {
      int ret = 0;
      int i = k + 1;
      while (k) {
        ret += i * c0[k] - c1[k];
        k -= lowbit(k);
      }
      return ret;
    }
    
    void modify(int l, int r, int v) {
      add(l, v), add(r + 1, -v);  // 将区间加差分为两个前缀加
    }
    
    long long query(int l, int r) {
      return sum(r) - sum(l - 1);
    }
}
\end{minted}

\subsection{三维前缀和}
\par \noindent 树状数组维护三维前缀和
\par $$\sum_{i=1}^{n}\sum_{j=1}^i\sum_{k=1}^jd_k=\frac{1}{2}[(n^2+3n+2)\sum_{i=1}^nd_i-(2n+3)\sum_{i=1}^ni·d_i+\sum_{i=1}^ni^2·d_i]$$
\par \noindent 由此也成功转化成三个一阶前缀和问题,如果继续推下去,会发现 $n$ 个树状数组就可以解决 $n$ 阶前缀和问题
\begin{minted}{c++}
namespace Fenwick {
    const int maxn = 1100;
    int c0[maxn], c1[maxn], c2[maxn];
    void add(int k, int v)
    {
        int i = k * v, ii = 1ll * k * k * v;
        while (k < maxn)
        {
            c0[k] += v;
            c1[k] += i;
            c2[k] += ii;
            k += k & -k;
        }
    }
    int sum(int k)
    {
        int ans = 0;
        int k1 = 1ll * k * k + 3 * k + 2, k2 = 2 * k + 3;
        while (k)
        {
            ans += k1 * c0[k] - k2 * c1[k] + c2[k];
            k -= k & -k;
        }
        return ans >> 1;
    }
    void modify(int l, int r, int v)
    {
        add(l, v);
        add(r + 1, -v);
    }
}
\end{minted}

\subsection{二维树状数组(矩阵求和与修改)}

\begin{minted}{c++}
namespace Fenwick_2D {
    int c0[maxn][maxn], c1[maxn][maxn], c2[maxn][maxn], c3[maxn][maxn];
    int n, m;
    inline int lowbit(int x) {
        return x & (-x);
    }
    void add(int x, int y, int v) {
        int i = x;
        while (i <= n) {
            int j = y;
            while (j <= m) {
                c0[i][j] += v;
                c1[i][j] += v * x;
                c2[i][j] += v * y;
                c3[i][j] += v * x * y;
                j += lowbit(j);
            }
            i += lowbit(i);
        }
    }
    int sum(int x, int y) {
        int ret = 0, i = x;
        while (i) {
            int j = y;
            while (j) {
                ret += c0[i][j] * (x + 1) * (y + 1);
                ret -= c1[i][j] * (y + 1);
                ret -= c2[i][j] * (x + 1);
                ret += c3[i][j];
                j -= lowbit(j);
            }
            i -= lowbit(i);
        }
        return ret;
    }
        /* 在矩形 (x0, y0),(x1, y1) 上加上 v */
    void change(int x0, int y0, int x1, int y1, int v) {
        add(x0, y0, v);
        add(x1 + 1, y0, -v);
        add(x0, y1 + 1, -v);
        add(x1 + 1, y1 + 1, v);
    }
    /* 查询矩形 (x0, y0),(x1, y1) 所有元素的和 */
    int query(int x0, int y0, int x1, int y1) {
        return sum(x1, y1) - sum(x0 - 1, y1) - sum(x1, y0 - 1) + sum(x0 - 1, y0 - 1);
    }
}using namespace Fenwick_2D;
\end{minted}
\subsection{三维树状数组}
\begin{minted}{c++}
inline int lowbit(int x) { return x & -x; }
void update(int x, int y, int z, int d) {
    for (int i = x; i <= n; i += lowbit(i))
        for (int j = y; j <= n; j += lowbit(j))
            for (int k = z; k <= n; k += lowbit(k))
                c[i][j][k] += d;
}
long long query(int x, int y, int z) {
    long long ret = 0;
    for (int i = x; i > 0; i -= lowbit(i))
        for (int j = y; j > 0; j -= lowbit(j))
            for (int k = z; k > 0; k -= lowbit(k))
                ret += c[i][j][k];
    return ret;
}
long long solve(int x0, int y0, int z0, int x1, int y1, int z1) {
    return    query(x1, y1, z1)
            - query(x1, y1, z0 - 1)
            - query(x1, y0 - 1, z1)
            - query(x0 - 1, y1, z1)
            + query(x1, y0 - 1, z0 - 1)
            + query(x0 - 1, y1, z0 - 1)
            + query(x0 - 1, y0 - 1, z1)
            - query(x0 - 1, y0 - 1, z0 - 1);
\end{minted}