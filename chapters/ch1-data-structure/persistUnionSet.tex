\section{可持久化并查集}
\begin{tcolorbox}
给定 $n$ 个集合,第 $i$ 个集合内初始状态下只有一个数,为 $i$。
有 $m$ 次操作。操作分为 3 种:
\begin{itemize}
\item 1 $a$ $b$ 合并 $a,b$ 所在集合
\item 2 $k$ 回到第 $k$ 次操作(执行三种操作中的任意一种都记为一次操作)之后的状态
\item 3 $a$ $b$ 询问 $a,b$ 是否属于同一集合,如果是则输出 1,否则输出 0。

\end{itemize}
\end{tcolorbox}
\begin{minted}{c++}
#include<bits/stdc++.h>

using namespace std;

using pii=pair<int,int>;
constexpr int N(200005);
int n,m;

struct Array
{
    struct Segment
    {
        int l,r,v;
    }t[N*80];
    int rt[N],cnt;
    void build(int &u,int l,int r,int c)
    {
        t[u=++cnt]={0,0,0};
        if(l==r) 
        {
            if(c==1) t[u].v=l;
            else if(c==2) t[u].v=1;
            return;
        }
        int mid=l+r>>1;
        build(t[u].l,l,mid,c),build(t[u].r,mid+1,r,c);
    }
    void update(int &u,int pre,int l,int r,int pos,int v)
    {
        t[u=++cnt]=t[pre];
        if(l==r) return t[u].v=v,void();
        int mid=l+r>>1;
        if(pos<=mid) 
            update(t[u].l,t[pre].l,l,mid,pos,v);
        else
            update(t[u].r,t[pre].r,mid+1,r,pos,v);
    }
    int query(int u,int l,int r,int pos)
    {
        if(l==r) return t[u].v;
        int mid=l+r>>1;
        if(pos<=mid) 
            return query(t[u].l,l,mid,pos);
        else
            return query(t[u].r,mid+1,r,pos);
    }
    void clear(){cnt=0;}
    int inline get(int u,int pos){
        return query(rt[u],1,n,pos);
    }
    void inline ins(int u,int pos,int v)
    {
        update(rt[u],rt[u],1,n,pos,v);
    }
}sz,fa;
int find(int u,int x)
{
    int f=fa.get(u,x);
    return x==f?x:find(u,f);
}
bool inline merge(int v, int a, int b) {
    a=find(v,a),b=find(v,b);
    if(a==b) return false;
    int sa=sz.get(v,a),sb=sz.get(v,b);
    if (sa>sb) swap(a,b),swap(sa,sb);
    fa.ins(v,a,b);sz.ins(v,b,sa+sb);
    return true;
}
int main()
{
    ios::sync_with_stdio(false);cin.tie(nullptr);cout.tie(nullptr);

    cin>>n>>m;
    fa.build(fa.rt[0],1,n,1);
    sz.build(sz.rt[0],1,n,2);

    int op,x,y;
    for(int i=1;i<=m;i++) 
    {
        cin>>op>>x;
        if(op==1)
        {
            cin>>y;
            fa.rt[i]=fa.rt[i-1];
            sz.rt[i]=sz.rt[i-1];
            merge(i,x,y);
        }
        else if(op==2)
        {
            fa.rt[i]=fa.rt[x];
            sz.rt[i]=sz.rt[x];
        }
        else 
        {
            cin>>y;
            int fx=find(i-1,x),fy=find(i-1,y);
            cout<<int(fx==fy)<<'\n';
            fa.rt[i]=fa.rt[i-1];
            sz.rt[i]=sz.rt[i-1];
        }
    }
    return 0;
}
\end{minted}
