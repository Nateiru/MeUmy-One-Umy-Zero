\section{放弃珂朵莉树}
\par \noindent 放弃珂朵莉树: (CF1638E,学习自jiangly) 维护同色段,可以直接用线段树,也可以用 std::map<int, Info>,其中 key 为段的左端点,左闭右开这样 next 就是右端点,value 为段的信息,这种方式非常好写。每次进行区间染色 $[l,r)$ 的时候,先把 $l,r$ 两点加入map,然后就可以直接循环删除中间的,然后处理修改。
\begin{tcolorbox}
\par \noindent 给定一个长度为 $n$ 的序列,初始时所有元素的值为 0 ,颜色为 1。你需要实现以下三种操作:
\begin{itemize}
\item Color $l$ $r$ $c$ :把 $[l,r]$这段的元素颜色改为 $c$
\item Add $c$ $x$:把所有颜色为 $c$ 的元素值都加 $x$
\item Query $i$:输出元素 $i$ 的值
\end{itemize}
\end{tcolorbox}
\begin{minted}{c++}
#include <bits/stdc++.h>
using namespace std;
using ll = long long;


const int N = 1000010;
int n, m;
// map<key,value>  key  表示区间左端点
//                 next 表示区间右端点 [it, next(it))
map<int, int> s;

map<int, int>::iterator split(int x) {
    auto it = prev(s.upper_bound(x));

    if (it->first == x)
        return it;

    return s.emplace(x, it->second).first;
}
// 区间修改单点查询树状数组
namespace Fenwick{
    ll t[N];
    void add(int k, ll x) { for (; k <= n; k += (k & -k)) t[k] += x;}
    ll sum(int k) { ll v = 0; for (; k; k -= (k & -k)) v += t[k]; return v;}
    void update(int l, int r, ll x) { add(l, x); add(r, -x); }
}using namespace Fenwick;

ll tag[N]; // tag[i]颜色是i的懒标记

int main() {
    ios::sync_with_stdio(false);
    cin.tie(nullptr);
    cout.tie(nullptr);

    cin >> n >> m;
    // 初始只有一个区间 [1, n+1) 颜色是 1 维护的每一段区间颜色都是相同的
    s[1] = 1;
    s[n + 1] = 0;

    char op[10];
    while (m--) {
        cin >> op;

        if (*op == 'C') {
            int l, r, c;
            cin >> l >> r >> c;
            r++;// 找到 [l, r + 1) 区间
            map<int, int>::iterator it = split(l);
            map<int, int>::iterator ti = split(r);
            
            // 边打标机边合并(erase)区间 最终得到 [l, r + 1)
            for (; it != ti; it = s.erase(it)) 
                update(it->first, next(it)->first, tag[it->second] - tag[c]);
            
            s[l] = c; 
        } else if (*op == 'A') {
            int c, x;
            cin >> c >> x;
            tag[c] += x;
        } else {
            int i;
            cin >> i;
            map<int, int>::iterator it = split(i);
            cout << sum(i) + tag[it->second] << '\n';
        }
    }

    return 0;
}
\end{minted}