\section{Link Cut Tree}

\par \noindent \textbf{实链剖分:}
~\\
\par \noindent 对于一个点连向它所有儿子的边,我们自己选择一条边进行剖分,我们称被选择的边为实边,其他边则为虚边。对于实边,我们称它所连接的儿子为实儿子。对于一条由实边组成的链,我们同样称之为实链。请记住我们选择实链剖分的最重要的原因:它是我们选择的,灵活且可变。正是它的这种灵活可变性,我们采用 Splay Tree 来维护这些实链。
~\\
\par \noindent LCT 理解成用一些 Splay 来维护动态的树链剖分,以期实现动态树上的区间操作。对于每条实链,我们建一个 Splay 来维护整个链区间的信息。
~\\
\par \noindent \textbf{辅助树:}
~\\
\par \noindent 对于一个点连向它所有儿子的边,我们自己选择一条边进行剖分,我们称被选择的边为实边,其他边则为虚边。对于实边,我们称它所连接的儿子为实儿子。对于一条由实边组成的链,我们同样称之为实链。请记住我们选择实链剖分的最重要的原因:它是我们选择的,灵活且可变。正是它的这种灵活可变性,我们采用 Splay Tree 来维护这些实链。
\begin{itemize}
\item 辅助树由多棵 Splay 组成,每棵 Splay 维护原树中的一条(极大)路径(实链),且中序遍历这棵 Splay 得到的点序列,从前到后对应原树\textbf{从上到下}的一条路径。换句话说,Splay 中的点的排序权值是其在原树中的深度。我们不会显式地指定权值。
\item 原树每个节点与辅助树的 Splay 节点一一对应。
\item 辅助树的各棵 Splay 之间并不是独立的。每棵 Splay 的根节点的父亲节点\textbf{本应}是空,但在 LCT 中每棵 Splay 的根节点的父亲节点指向\textbf{原树}中\textbf{这条(实)链} 的父亲节点(即链最顶端的点的父亲节点)。这类父亲链接与通常 Splay 的父亲链接区别在于儿子认父亲,而父亲不认儿子,对应原树的一条 \textbf{虚边}。因此,每个连通块恰好有一个点的父亲节点为空。
\item 由于辅助树的以上性质,我们维护任何操作都不需要维护原树,辅助树可以在任何情况下拿出一个唯一的原树,我们只需要维护辅助树即可。
\end{itemize}

\par \noindent \textbf{考虑原树和辅助树的结构关系:}
\begin{itemize}
\item 原树中的实链 : 在辅助树中节点都在一棵 Splay 中。辅助树中每棵Splay通过后继和前驱维护原树中的父子关系!
\item 原树中的虚链 : 在辅助树中,每棵 Splay 的根节点的父亲节点\textbf{本应}是空,通过根节点的父亲维护虚边。子节点所在 Splay 的 Father 指向父节点,但是父节点的两个儿子都不指向子节点。
\end{itemize}

\par \noindent \textbf{认父不认子:}
~\\
\par \noindent 边分为实边和虚边,实边包含在Splay中,而虚边总是由一棵Splay指向另一个节点(指向该Splay中中序遍历最靠前的点在原树中的父亲)。
~\\
\par \noindent 当某点在原树中有多个儿子时,只能向其中一个儿子拉一条实链(只认一个儿子),而其它儿子是不能在这个Splay中的。
~\\
\par \noindent 那么为了保持树的形状,我们要让到其它儿子的边变为虚边,由\textbf{对应儿子所属的Splay的根节点的父亲}指向该点,而从该点并不能直接访问该儿子(认父不认子)。
~\\
\par \noindent \textbf{总结:}
~\\
\par \noindent 原树被剖分成一些实链,每条实链通过虚边连接。每条实链用一颗Spaly维护,中序遍历从前到后对应原树\textbf{从上到下}的一条路径。每棵 Splay 的根节点的父亲节点\textbf{本应}是空,通过该根节点的父亲维护虚边:每棵 Splay 的根节点的父亲节点指向\textbf{原树}中\textbf{这条(实)链} 的父亲节点(即链最顶端的点的父亲节点)。把一些Splay连接起来构成辅助树。

\subsection{模板}
\begin{minted}{c++}
//若要修改一个点的点权,应当先将其 splay 到根,然后修改,最后还要调用 pushup 维护。
namespace lct {
    int ch[maxn][2], fa[maxn], stk[maxn], rev[maxn];
    int sz[maxn];
    void init() { //初始化 link-cut-tree
        memset(ch, 0, sizeof(ch));
        memset(fa, 0, sizeof(fa));
        memset(rev, 0, sizeof(rev));
        memset(sz, 0, sizeof(sz));
    }
    inline bool son(int x) {
        return ch[fa[x]][1] == x;
    }
    inline bool isroot(int x) {
        return ch[fa[x]][1] != x && ch[fa[x]][0] != x;
    }
    inline void reverse(int x) { //给结点 x 打上反转标记
        swap(ch[x][1], ch[x][0]);
        rev[x] ^= 1;
    }
    inline void pushup(int x) {
        sz[x] = sz[ch[x][0]] + sz[ch[x][1]] + 1;
    }
    inline void pushdown(int x) {
        if (rev[x]) {
            reverse(ch[x][0]);
            reverse(ch[x][1]);
            rev[x] = 0;
        }
    }
    void rotate(int x) {
        int y = fa[x], z = fa[y], c = son(x);
    
        if (!isroot(y))
            ch[z][son(y)] = x;
    
        fa[x] = z;
        ch[y][c] = ch[x][!c];
        fa[ch[y][c]] = y;
        ch[x][!c] = y;
        fa[y] = x;
        pushup(y);
    }
    void splay(int x) {
        int top = 0;
        stk[++top] = x;
    
        for (int i = x; !isroot(i); i = fa[i])
            stk[++top] = fa[i];
    
        while (top)
            pushdown(stk[top--]);
    
        for (int y = fa[x]; !isroot(x); rotate(x), y = fa[x])
            if (!isroot(y))
                son(x) ^ son(y) ? rotate(x) : rotate(y);
    
        pushup(x);
    }
    void access(int x) { // 建立从根到 x 的路径
        for (int y = 0; x; y = x, x = fa[x]) {
            splay(x);
            ch[x][1] = y;
            pushup(x);
        }
    }
    void makeroot(int x) { //将 x 变为树的新的根结点
        access(x);
        splay(x);
        reverse(x);
    }
    int findroot(int x) { //返回 x 所在树的根结点
        access(x);
        splay(x);
    
        while (ch[x][0])
            pushdown(x), x = ch[x][0];
    
        splay(x);
        return x;
    }
    void split(int x, int y) { //提取出来 y 到 x 之间的路径,并将 y 作为根结点
        makeroot(x);
        access(y);
        splay(y);
    }
    void cut(int x, int y) { //切断 x 与 y 相连的边
        makeroot(x);         //将 x 置为整棵树的根
    
        if (findroot(y) == x && fa[y] == x && !ch[y][0]) {
            fa[y] = ch[x][1] = 0;
            pushup(x);
        }
    }
    void link(int x, int y) { //连接 x 与 y
        makeroot(x);
    
        if (findroot(y) != x)
            fa[x] = y;
    }
}
\end{minted}


\subsection{LCT 动态维护直径}
\begin{tcolorbox}
\par \noindent 一开始有 $n$ 个点的无边无向图,接下来有 $q$ 次操作,每次操作分为以下两种:
\begin{itemize}
\item 1 $u$ $v$:将 $u$ 和 $v$ 连边,保证 $u$ 和 $v$ 不连通。
\item 2 $u$:询问 $u$ 能到达的最远的点与 $u$ 的距离。
\end{itemize}
\end{tcolorbox}
\par \noindent 两颗子树合并后的直径端点只有六种情况。六种情况讨论一下即可维护合并后树的直径。

\begin{minted}{c++}
//若要修改一个点的点权,应当先将其 splay 到根,然后修改,最后还要调用 pushup 维护。
using namespace lct;

const int N = 300005;

int dis(int x, int y) {
    split(x, y);
    return sz[y] - 1;
}
struct dia {
    int u, v, w;
    bool operator<(const dia &o)const {
        return w < o.w;
    }
    dia operator +(const dia &o)const {
        dia ret = max(*this, o);// 原来树的直径
        // 两条直径端点组合构成新的直径
        ret = max(ret, {u, o.u, dis(u, o.u)});
        ret = max(ret, {u, o.v, dis(u, o.v)});
        ret = max(ret, {v, o.u, dis(v, o.u)});
        ret = max(ret, {v, o.v, dis(v, o.v)});
        return ret;
    }
} d[N];
int dsu[N];
int find(int x) {
    return x == dsu[x] ? x : dsu[x] = find(dsu[x]);
}
void merge(int x, int y) {
    link(x, y);
    x = find(x), y = find(y);
    dsu[x] = y;
    d[y] = d[x] + d[y];
}
int n, q;
int type;

int main() {
    ios::sync_with_stdio(false); cin.tie(nullptr); cout.tie(nullptr);
    init();
    cin >> type >> n >> q;

    for (int i = 1; i <= n; i++) dsu[i] = i, sz[i] = 1, d[i] = {i, i, 0};

    int lastans = 0;// 强制在线

    while (q--) {
        int op;
        cin >> op;
        if (op == 1) {
            int u, v;
            cin >> u >> v;
            if (type) u ^= lastans, v ^= lastans;
            merge(u, v);
        } else {
            int x;
            cin >> x;
            if (type) x ^= lastans;
            auto [u, v, w] = d[find(x)];
            lastans = max(dis(x, u), dis(x, v));
            cout << lastans << '\n';
        }
    }
    return 0;
}
\end{minted}
\par \noindent LCT可以维护森林的树上距离,如果树的形态在询问前可以确定,那么可以使用O1求LCA从而求出树上距离,进而可以离线维护直径。


\subsection{LCT 维护子树大小}
\par \noindent lct在动态连边和删边方面比较有优势,但是在维护子树信息方面又没有树链剖分那么方便。

\par \noindent 要维护虚子树信息 在lct原来的模板上有三个地方需要改。

\begin{itemize}
\item pushup函数:总子树大小显然是实子树大小+虚子树大小
\item access函数:在进行access的过程,x的虚子树产生了变化,本来是y,后来变成了x现在的右儿子。
\item link函数:在连接两个点的时候,(x连y)我们把x连做y的虚儿子,显然y的虚子树需要加上x的大小 另外需要注意的是 必须把y结点splay到最上面才能保证更新的正确性(类似splay的更新原理)
\end{itemize}
\begin{minted}{c++}
inline void pushup(int x) {
        sz[x] = sz[ch[x][0]] + sz[ch[x][1]] + 1 + vsz[x]; // 虚儿子的大小
    }
void access(int x) { // 建立从根到 x 的(实边)路径
        for (int y = 0; x; y = x, x = fa[x]) {
            splay(x);
            vsz[x] += sz[ch[x][1]];  // 虚儿子变为实儿子
            ch[x][1] = y;
            vsz[x] -= sz[ch[x][1]];  // 实儿子变为虚儿子
            pushup(x);
        }
    }
void link(int x, int y) { //连接 x 与 y
        makeroot(x);        // x 是原树的根节点,同时 x 也是所在Splay的根节点        
        makeroot(y);
           fa[x] = y;            // x 是 y 的一个虚儿子
           vsz[y] += sz[x];
        pushup(y);
    }
\end{minted}