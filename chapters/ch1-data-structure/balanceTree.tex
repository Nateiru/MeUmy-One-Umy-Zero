\section{平衡树}

\begin{itemize}
\item 维护值域:支持前驱、后继、排名.......
\item 维护序列:区间翻转、区间平移.......
\end{itemize}
\subsection{Splay 【模板】普通平衡树(维护值域)}
\begin{minted}{c++}
#include<bits/stdc++.h>
using namespace std;
const int maxn = 110000;
//若要修改一个点的点权,应当先将其 splay 到根,然后修改,最后还要调用 pushup 维护。
//调用完 splay 之后根结点会改变,应该用 splay 的返回值更新根结点。

namespace splay_tree {
    int ch[maxn][2], fa[maxn], stk[maxn], rev[maxn], sz[maxn], key[maxn], tot;
    int rt, cnt[maxn];
    void init() {
        tot = rt = 0;
    }
    int newnode(int val) {
        int x = ++tot;
        ch[x][0] = ch[x][1] = fa[x] = rev[x] = 0;
        sz[x] = cnt[x] = 1;
        key[x] = val;
        return x;
    }
    void clear(int x) {
        ch[x][0] = ch[x][1] = fa[x] = key[x] = sz[x] = cnt[x] = rev[x] = 0;
    }
    inline bool son(int x) {// x是fa的哪个儿子 0/1 左/右
        return ch[fa[x]][1] == x;
    }
    inline void pushup(int x) {
        sz[x] = sz[ch[x][0]] + sz[ch[x][1]] + cnt[x];
    }
    inline void pushdown(int x) {
        if (rev[x]) {
            rev[x] = 0;
            swap(ch[x][0], ch[x][1]);
            rev[ch[x][0]] ^= 1;
            rev[ch[x][1]] ^= 1; 
        }
    }
    void rotate(int x) { //左旋和右旋
        int y = fa[x], z = fa[y], c = son(x);
        if (fa[y])
            ch[z][son(y)] = x;
        fa[x] = z;

        ch[y][c] = ch[x][!c];fa[ch[y][c]] = y;
        ch[x][!c] = y;fa[y] = x;
        pushup(y);
    }
    void ascend(int x) { // 将 x 反转到根
        for (int y = fa[x]; y; rotate(x), y = fa[x]) 
            if (fa[y]) son(x) ^ son(y) ? rotate(x) : rotate(y);
        pushup(x);
        rt = x;
    }
    int splay(int x) {   // 没有 pushdown 操作时,可以直接用 ascend 替换 splay
        int top = 0;     // 每访问一个节点 x 后都要强制将其旋转到根节点
        for (int i = x; i; i = fa[i])
            stk[++top] = i;
        while (top)
            pushdown(stk[top--]);
        ascend(x);
        return x;
    }
    int splay(int x, int k) { //将以 x 为根的子树中的第 k 个结点旋转到根结点
        while (pushdown(x), k != sz[ch[x][0]] + 1) {
            if (k <= sz[ch[x][0]])
                x = ch[x][0];
            else
                k -= sz[ch[x][0]] + 1, x = ch[x][1];
        }
        if (x) ascend(x);
        return x;
    }
    void ins(int k) { // 插入 k 数
        if(!rt) {
            rt = newnode(k);
            pushup(rt);
            return;
        }
        int cur = rt, f = 0;
        while (1) {
            if (key[cur] == k) {
                cnt[cur] ++;
                pushup(cur);
                pushup(f);
                splay(cur);
                break;
            }
            f = cur;
            cur = ch[cur][key[cur] < k];
            if(!cur) {
                int x = newnode(k);
                fa[x] = f;
                ch[f][key[f] < k] = x;
                pushup(x);
                pushup(f);
                splay(x);
                break;
            }
        }
    }
    int rk(int k) { // k的排名 排名定义为比当前数小的数的个数 +1
        int ret = 0, cur = rt;
        while (1) {
            if (k < key[cur]) 
                cur = ch[cur][0];
            else {
                ret += sz[ch[cur][0]];
                if (k == key[cur]) {
                    splay(cur);
                    return ret + 1;
                }
                ret += cnt[cur];
                cur = ch[cur][1];
            }
        }
    }
    int kth(int k) { // 第k大
        int cur = rt;
        while (1) {
            if (ch[cur][0] && k <= sz[ch[cur][0]]) 
                cur = ch[cur][0];
            else {
                k -= cnt[cur] + sz[ch[cur][0]];
                if (k <= 0) {
                  splay(cur);
                  return key[cur];
                }
                cur = ch[cur][1];
            }
        }
    }
    int pre() { // 根节点前驱
        int cur = ch[rt][0];
        if (!cur) return cur;
        while (ch[cur][1]) cur = ch[cur][1];
        splay(cur);
        return cur;
    }
    int nxt() { // 根节点后继
        int cur = ch[rt][1];
        if (!cur) return cur;
        while (ch[cur][0]) cur = ch[cur][0];
        splay(cur);
        return cur;
    }
    void del(int k) {  // 删除 x 数(若有多个相同的数,因只删除一个)
        rk(k);
        if (cnt[rt] > 1) {
            cnt[rt]--;
            pushup(rt);
            return;
        }
        if (!ch[rt][0] && !ch[rt][1]) {
            clear(rt);
            rt = 0;
            return;
        }
        if (!ch[rt][0]) {
            int cur = rt;
            rt = ch[rt][1];
            fa[rt] = 0;
            clear(cur);
            return;
        }
        if (!ch[rt][1]) {
            int cur = rt;
            rt = ch[rt][0];
            fa[rt] = 0;
            clear(cur);
            return;
        }
        int cur = rt, x = pre();
        fa[ch[cur][1]] = x;
        ch[x][1] = ch[cur][1];
        clear(cur);
        pushup(rt);
    }
}
using namespace splay_tree;


int main() {

    init();
    int n, opt, x;
    for (scanf("%d", &n); n; --n) {
        scanf("%d%d", &opt, &x);
        if (opt == 1)
            ins(x);
        else if (opt == 2)
            del(x);
        else if (opt == 3)
            printf("%d\n", rk(x));
        else if (opt == 4)
            printf("%d\n", kth(x));
        else if (opt == 5)
            ins(x), printf("%d\n", key[pre()]), del(x);
        else
            ins(x), printf("%d\n", key[nxt()]), del(x);
    }
    return 0;
}
\end{minted}

\subsection{Splay 【模板】文艺平衡树(维护序列)}
\begin{minted}{c++}
#include<bits/stdc++.h>
using namespace std;
const int maxn = 100005;
//若要修改一个点的点权,应当先将其 splay 到根,然后修改,最后还要调用 pushup 维护。
//调用完 splay 之后根结点会改变,应该用 splay 的返回值更新根结点。

namespace splay_tree {
    int ch[maxn][2], fa[maxn], stk[maxn], rev[maxn], sz[maxn], key[maxn], cnt;
    void init() {
        cnt = 0;
    }
    int newnode(int val) {
        int x = ++cnt;
        ch[x][0] = ch[x][1] = fa[x] = rev[x] = 0;
        sz[x] = 1;
        key[x] = val;
        return x;
    }
    inline bool son(int x) {// x是fa的哪个儿子 0/1 左/右
        return ch[fa[x]][1] == x;
    }
    inline void pushup(int x) {
        sz[x] = sz[ch[x][0]] + sz[ch[x][1]] + 1;
    }
    inline void pushdown(int x) {
        if (rev[x]) {
            rev[x] = 0;
            swap(ch[x][0], ch[x][1]);
            rev[ch[x][0]] ^= 1;
            rev[ch[x][1]] ^= 1; 
        }
    }
    void rotate(int x) { //左旋和右旋
        int y = fa[x], z = fa[y], c = son(x);
        if (fa[y])
            ch[z][son(y)] = x;
        fa[x] = z;
        
        ch[y][c] = ch[x][!c];fa[ch[y][c]] = y;
        ch[x][!c] = y;fa[y] = x;
        pushup(x);pushup(y);
    }
    void ascend(int x) { // 将 x 反转到根
        for (int y = fa[x]; y; rotate(x), y = fa[x]) 
            if (fa[y]) son(x) ^ son(y) ? rotate(x) : rotate(y);
        pushup(x);
    }
    int splay(int x) {   // 没有 pushdown 操作时,可以直接用 ascend 替换 splay
        int top = 0;
        for (int i = x; i; i = fa[i])
            stk[++top] = i;
        while (top)
            pushdown(stk[top--]);
        ascend(x);
        return x;
    }
    int splay(int x, int k) { //将以 x 为根的子树中的第 k 个结点旋转到根结点
        while (pushdown(x), k != sz[ch[x][0]] + 1) {
            if (k <= sz[ch[x][0]])
                x = ch[x][0];
            else
                k -= sz[ch[x][0]] + 1, x = ch[x][1];
        }
        if (x) ascend(x);
        return x;
    }
    template<typename ...T> int merge(int x, int y, T... args) {
        if constexpr (sizeof...(args) == 0) {
            if (x == 0) return y; //swap(x, y);
            x = splay(x, sz[x]);
            ch[x][1] = y; fa[y] = x;
            pushup(x);
            return x;
        }
        else {
            return merge(merge(x, y), args...);
        } 
    }
    pair<int, int> split(int x, int pos) { //分成两个区间 [1, pos - 1] 和 [pos, n]
        if (pos == sz[x] + 1)
            return make_pair(x, 0);
        x = splay(x, pos);    // 找到x子树中的第pos个数
        int y = ch[x][0];
        fa[y] = ch[x][0] = 0; // 断掉边
        pushup(x);
        return make_pair(y, x);
    }
    // [1, L-1] [L, R] [R+1, n]
    auto extract(int x, int L, int R) {
        auto [left, y] = split(x, L);
        auto [mid, right] = split(y, R - L + 2);
        return make_tuple(left, mid, right);
    }
    void traverse(int x) { //中序遍历
        if (x != 0) {
            pushdown(x);
            traverse(ch[x][0]);
            printf("%d ", key[x]);
            //printf("%d (left: %d, right: %d) sz(%d) key(%d)\n", x, ch[x][0], ch[x][1], sz[x],key[x]);
            traverse(ch[x][1]);
        }
    }
}
using namespace splay_tree;


int main() {
    // init();
    // int nd[50], rt = 0;
    // for (int i = 1; i <= 10; ++i) {
    //     nd[i] = newnode(i);
    //     rt = merge(rt, nd[i]);
    // }
    // traverse(get<1>(extract(rt, 3, 10))); printf("\n");
    
    init();
    int n, m, rt=0;cin>>n>>m;
    for(int i = 1; i <= n; i++) rt = merge(rt, newnode(i));
    while (m--) {
        int l, r; 
        cin >> l >> r;
        auto t = extract(rt, l, r);
        #define X(x) get<0>(x)
        #define Y(x) get<1>(x)
        #define Z(x) get<2>(x)
        rev[Y(t)] ^= 1;
        rt = merge(X(t),Y(t),Z(t));
    }
    traverse(rt);
    return 0;
}
\end{minted}

\subsection{无旋 Treap【模板】普通平衡树(维护值域)}
\par \noindent treap 的每个结点上除了关键字val之后,还要额外储存一个值 key。
\begin{itemize}
\item val 满足BST性质
\item key满足小根堆性质
\end{itemize}
\par \noindent 而 key 是每个结点建立时随机生成的,因此 treap 是\textbf{期望平衡(即高度=$\log n$)}的。
~\\
\par \noindent 和splay一样,treap也有 2 种主要用法:\textbf{维护序列}和\textbf{维护值域}。
~\\
\par \noindent split可以有按值分裂,按个数(排名)分裂。分别对应维护值域和维护序列。
\begin{minted}{c++}
#include<bits/stdc++.h>
using namespace std;
using ll = long long;

const int seed = []() {
    random_device rds;
    return rds();
}();
mt19937 rd(seed);
// 无旋Treap维护序列(不维护值域)
// 不支持 newnode() 需要提前标号认为多颗Treap直接合并

struct Treap {
    struct T {
        const int key;      // 随机值 满足堆的性质
        int ls, rs, sz;     // 基本值数量 
        int val;            // 维护序列的值
        T() : key(rd()) {
            ls = rs = 0;
            sz = 0;
            val = 0;
        }
    };
    vector<T> t;
    int rt, cnt;
    
    Treap(int n) : t(n + 1), rt(0), cnt(0) {
    }
    
    int newnode(int x) {
        t[++cnt].val = x;
        t[cnt].sz = 1;
        return cnt;
    }
    void pushup(int u) {
        t[u].sz = t[t[u].ls].sz + t[t[u].rs].sz + 1;
    }
    // 两棵树合并
    int merge(int x, int y) {
        if (!x || !y) {
            return x | y;
        }
        int u = 0;
        if (t[x].key < t[y].key) {
            u = x, t[u].rs = merge(t[u].rs, y);
        } else {
            u = y, t[u].ls = merge(x, t[u].ls);
        }
        pushup(u);
        return u;
    }
    // 按val分裂 分裂出 <= val 的部分
    void split(int u, int val, int &x, int &y) {
        if (!u) {
            x = y = 0;
            return;
        }
        if (t[u].val <= val) {
            x = u;
            split(t[u].rs, val, t[x].rs, y);
        }
        else {
            y = u;
            split(t[u].ls, val, x, t[y].ls);
        }
        pushup(u);
    }
    // 分裂出 <= val 的部分
    pair<int, int> split(int u, int val) {
        int x, y;
        split(u, val, x, y);
        return make_pair(x, y);
    }
    void ins(int val) {
        auto [x, y] = split(rt, val);
        rt = merge(merge(x, newnode(val)), y);
    }
    void del(int val) {
        auto [w, z] = split(rt, val);
        auto [x, y] = split(w, val - 1);
        y = merge(t[y].ls, t[y].rs); // 删去一个点相当于合并
        rt = merge(merge(x, y), z);
    }
    // val的排名
    int getrank(int val) {
        auto [x, y] = split(rt, val - 1);
        int cnt = t[x].sz + 1;
        rt = merge(x, y);
        return cnt;
    }
    // 排名时k的值
    int getval(int k) {
        int u = rt;
        while(u) {
            if(t[t[u].ls].sz + 1 == k) break;
            else if(t[t[u].ls].sz >= k) u = t[u].ls;
            else {
                k -= t[t[u].ls].sz + 1;
                u = t[u].rs;
            }
        }
        return t[u].val;
    }
    int pre(int val) {
        auto [x, y] = split(rt, val - 1);
        int u = x;
        while(t[u].rs) u = t[u].rs;
        rt = merge(x, y);
        return t[u].val;
    }
    int suc(int val) {
        auto [x, y] = split(rt, val);
        int u = y;
        while(t[u].ls) u = t[u].ls;
        rt = merge(x, y);
        return t[u].val;
    }
};

int main() {
    ios::sync_with_stdio(false);cin.tie(nullptr), cout.tie(nullptr);
    int n;
    cin >> n;
    Treap t(n + 1);
    while (n--) {
        int op, x;
        cin >> op >> x;
        if (op == 1) t.ins(x);
        else if (op == 2) t.del(x);
        else if (op == 3) cout << t.getrank(x) << '\n';
        else if (op == 4) cout << t.getval(x) << '\n';
        else if (op == 5) cout << t.pre(x) << '\n';
        else cout << t.suc(x) << '\n';
    }
    return 0;
}
\end{minted}

\subsection{无旋 Treap【模板】文艺平衡树(维护序列)}
\begin{minted}{c++}
#include<bits/stdc++.h>
using namespace std;
using ll = long long;

const int seed = []() {
    random_device rds;
    return rds();
}();
mt19937 rd(seed);
// 无旋Treap维护序列(不维护值域)
// 不支持 newnode() 需要提前标号认为多颗Treap直接合并

struct Treap {
    struct T {
        const int key;      // 随机值 满足堆的性质
        int ls, rs, fa, sz; // 基本值 同时维护父亲方便找根
        int val;            // 维护序列的值
        bool rev;           // 翻转标记
        T() : key(rd()) {
            ls = rs = fa = 0;
            sz = 1;
            val = 0;
            rev = 0;
        }
    };
    vector<T> t;
    
    Treap(int n) : t(n + 1) {
        t[0].sz = 0;
    }
    void pushup(int u) {
        t[u].sz = t[t[u].ls].sz + t[t[u].rs].sz + 1;
        t[t[u].ls].fa = t[t[u].rs].fa = u;
    }
    void pushdown(int u) {
        if(t[u].rev) {
            t[t[u].ls].rev ^= 1;
            t[t[u].rs].rev ^= 1;
            swap(t[u].ls, t[u].rs);
            t[u].rev = 0;
        }
    }
    // 两棵树合并
    int merge(int x, int y) {
        if (!x || !y) {
            return x | y;
        }
        int u = 0;
        if (t[x].key < t[y].key) {
            pushdown(x);
            u = x, t[u].rs = merge(t[u].rs, y);
        } else {
            pushdown(y);
            u = y, t[u].ls = merge(x, t[u].ls);
        }
        pushup(u);
        return u;
    }
    // 按子树个数分裂
    void split(int u, int k, int &x, int &y) {
        if (!u) {
            x = y = 0;
            return;
        }
        pushdown(u);
        int cnt = t[t[u].ls].sz + 1;
        if (cnt <= k) {
            x = u;
            split(t[u].rs, k - cnt, t[u].rs, y);
        } else {
            y = u;
            split(t[u].ls, k, x, t[u].ls);
        }
        pushup(u);
    }
    // 分裂出前 k 个数
    pair<int, int> split(int u, int k) {
        int x, y;
        split(u, k, x, y);
        t[x].fa = t[y].fa = 0;
        return make_pair(x, y);
    }
    void reverse(int &rt, int l, int r) {
        auto [x, y] = split(rt, l - 1);
        auto [z, w] = split(y, r - l + 1);
        t[z].rev ^= 1;
        rt = merge(merge(x, z), w);
    }
    void traverse(int u) { //中序遍历
        if (!u) return;
        pushdown(u);
        traverse(t[u].ls);
        printf("%d ", t[u].val);
        traverse(t[u].rs);
    }
    /**
     * 没有pushdown 可以使用下面两个操作
    **/
    int getroot(int u) {
        while (t[u].fa) {
            u = t[u].fa;
        }
        return u;
    }
    // 维护一个序列:u 节点前面树数的个数(不带pushdown)
    int getcnt(int u) {
        int cnt = t[t[u].ls].sz + 1;
        while (t[u].fa) {
            if (t[t[u].fa].rs == u) {
                cnt += t[t[t[u].fa].ls].sz + 1;
            }
            u = t[u].fa;
        }
        return cnt;
    }
};

int main() {
    ios::sync_with_stdio(false);cin.tie(nullptr), cout.tie(nullptr);
    
    int n, m;
    cin >> n >> m;
    
    int rt = 0;
    // 需要提前标号并赋予val认为多颗Treap直接合并
    Treap tr(n + 1);
    for (int i = 1; i <= n; i++) {
        tr.t[i].val = i;
        rt = tr.merge(rt, i);   
    }
    while(m--) {
        int l, r;
        cin >> l >> r;
        tr.reverse(rt, l, r);
    }
    tr.traverse(rt);
    
    return 0;
}
\end{minted}
