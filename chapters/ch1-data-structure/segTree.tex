\section{线段树}
\subsection{标记优先级}
\par \noindent 区间加、区间乘线段树

\begin{minted}{c++}
// sum = sum * mul + add * len
// add = add * mul + add
// mul = mul 
void puttag(int u, int add , int mul){
    t[u].sum = ((ll)t[u].sum * mul + (ll)(t[u].r - t[u].l + 1) * add) % mod;
    t[u].add = ((ll)t[u].add * mul + add) % mod;
    t[u].mul = (ll)t[u].mul * mul % mod;
}
void pushdown(int u){
    puttag(u << 1 , t[u].add , t[u].mul);
    puttag(u << 1 | 1 , t[u].add , t[u].mul);
    t[u].add = 0 , t[u].mul = 1;
}
// 另一种写法
// 优先级  mul > add
void mark_mul(int u, int mul) {
    t[u].sum = (ll)t[u].sum * mul % mod;
    t[u].add = (ll)t[u].add * mul % mod;
    t[u].mul = (ll)t[u].mul * mul % mod;
}
void mark_add(int u, int add) {
    t[u].sum = (t[u].sum + (ll)(t[u].r - t[u].l + 1) * add) % mod;
    t[u].add = ((ll)t[u].add + add) % mod;
}
void pushdown(int u){
    if(t[u].mul != 1) {
        mark_mul(u<<1, t[u].mul);
        mark_mul(u<<1|1, t[u].mul);
        t[u].mul = 1;
    }
    if(t[u].add) {
        mark_add(u<<1, t[u].add);
        mark_add(u<<1|1, t[u].add);
        t[u].add = 0;
    }
}
\end{minted}
\subsection{历史版本和}
\par \noindent 给定一个长度为 $n$ 的序列 $a_i$,现在有 $m$ 次操作,操作有两种:
\begin{tcolorbox}
\begin{itemize}
\item 给定区间 $[l,r]$ 和 $v$,将区间 $[l,r]$ 加上 $v$
\item 给定区间 $[l,r]$,求区间 $[l,r]$ 的历史版本和
\end{itemize}
\end{tcolorbox}
\par \noindent 用线段树维护区间和 $sum$、区间历史和 $sumh$,但是我们这里不引入时间的概念,我们将历史和 $sumh$ 看成是 $sum$ 乘上另一个标记 $tag$,而每段时间,我们手动更新 $tag$.
\par \noindent 这里认为 $tag$ 标记优先级高于加标记:
\begin{minted}{c++}
// U216697 线段树区间历史版本和 
// https://www.luogu.com.cn/problem/U216697
#include <cstdio>
#include <cstring>
#include <algorithm>
using namespace std;
using ll = long long;

const int  N = 100008;

int a[N], n, m;

struct Node {
    int len;
    ll sum, add;    //     区间和、 区间加标记
    ll sumh, addh;  // 区间历史和、历史和加标记
    ll tag;         // 次数乘上原来儿子上存的值
} t[N << 2];
void pushup(int u) {
    t[u].sum = t[u<<1].sum + t[u<<1|1].sum;
    t[u].sumh = t[u<<1].sumh + t[u<<1|1].sumh;
}
// 一起下传 考虑标记之间影响
void puttag(int u, ll add, ll addh, ll tag) {
    
    t[u].sumh += tag * t[u].sum + addh * t[u].len;
    t[u].sum += t[u].len * add;
    
    t[u].addh += tag * t[u].add + addh;
    t[u].tag += tag;
    t[u].add += add;
}
// 另一种写法
// 优先级  tag > addh > add
void puttag(int u, ll add, ll addh, ll tag) {
    t[u].sumh += tag * t[u].sum + addh * t[u].len;
    t[u].sum += t[u].len * add;
    
    t[u].addh += tag * t[u].add + addh;
    t[u].tag += tag;
    t[u].add += add;
}
void mark_tag(int u, ll tag) {
    t[u].sumh += tag * t[u].sum;
    t[u].addh += tag * t[u].add;
    t[u].tag += tag;
}
void mark_adh(int u, ll addh) {
    t[u].sumh += addh * t[u].len;
    t[u].addh += addh;
}
void mark_add(int u, ll add) {
    t[u].sum += t[u].len * add;
    t[u].add += add;
}
void pushdown(int u) {
    if (t[u].tag) {
        mark_tag(u<<1, t[u].tag);
        mark_tag(u<<1|1, t[u].tag);
        t[u].tag = 0;
    }
    if (t[u].addh) {
        mark_adh(u<<1, t[u].addh);
        mark_adh(u<<1|1, t[u].addh);
        t[u].addh = 0;
    }
    if (t[u].add) {
        mark_add(u<<1, t[u].add);
        mark_add(u<<1|1, t[u].add);
        t[u].add = 0;
    }
}
void build(int u, int l, int r) {
    t[u].len = r - l + 1;
    if (l == r) {
        t[u].sum = a[l];
        t[u].sumh = a[l];
        return;
    }
    int mid = (l + r) >> 1;
    build(u<<1, l, mid);
    build(u<<1|1, mid + 1, r);
    pushup(u);
}
void update(int u, int l, int r, int L, int R, int v) {
    if (l >= L && r <= R) {
        mark_add(u, v);
        return;
    }
    pushdown(u);
    int mid = (l + r) >> 1;
    if (L <= mid) update(u<<1, l, mid, L, R, v);
    if (R > mid) update(u<<1|1, mid + 1, r, L, R, v);
    pushup(u);
}
ll query(int u, int l, int r, int L, int R) {
    if (l >= L && r <= R) 
        return t[u].sumh;
    pushdown(u);
    int mid = (l + r) >> 1;
    ll v = 0;
    if (L <= mid) v += query(u<<1, l, mid, L, R);
    if (R > mid) v += query(u<<1|1, mid + 1, r, L, R);
    return v;
}
int main() {

    scanf("%d%d", &n, &m);
    for (int i = 1; i <= n; ++i) scanf("%d", &a[i]);
    build(1, 1, n);
    int op, x, y, l, r;
    for (int i = 1; i <= m; ++i) {
        scanf("%d", &op);
        if (op == 1) {
            scanf("%d%d%d", &l, &r, &x);
            update(1, 1, n, l, r, 1ll * x);
        }
        else {
            scanf("%d%d", &x, &y);
            printf("%lld\n", query(1, 1, n, x, y));
        }
        mark_tag(1, 1);
    }
    return 0;
}
\end{minted}
\begin{tcolorbox}
\par \noindent 给出一个长度为 $n$ 的序列 $a$,每次询问一个区间 $[l,r]$。询问有多少个子区间 $[i,j]$ 满足子区间内不同的数的数量是奇数。
\end{tcolorbox}
\begin{itemize}
\item 将询问离线,从左往右扫描右端点 $r$
\item 线段树每个点 $i$ 维护的是 $[i,r]$ 这个段区间不同数的个数是偶数/奇数
\item 那么对于区间询问 $[L, R]$ 得到的结果: 有多少形如 $[(L\to R), r]$ 区间不同数的个数是 偶数/奇数
\end{itemize}
\textbf{注意:$[(L\to R), r]$ 的区间左端点在 $[L,R]$ 右端点始终是目前扫描到的 $r$ }
~\\
\par \noindent 对于维护 $[L,R]$ 子区间 显然是维护所有答案的历史版本和!
~\\

\par \noindent \textbf{有时维护很多标记时,标记之间的关系非常混乱,考虑是否能用矩阵的形式代替。}
~\\
\par \noindent 考虑维护:

\begin{itemize}
\item 区间长度:$len=r-l+1$
\item 区间和:$sum$
\item 历史区间和:$sumh$
\end{itemize}

\par \noindent 考虑每次区间加 $val$,那么考虑各个数字的变化:

\begin{itemize}
\item 对于区间和变化量: $len×val$
\item 对于区间历史和变化量:$sum+len×val$
\end{itemize}

\par \noindent 于是有:

\begin{itemize}
\item $len = len$
\item $sum = sum + len × val$
\item $sumh = sumh + sum + len × val$
\end{itemize}

\par \noindent 把上面的式子整理一下,就可以用矩阵乘法维护(建议维护行向量用右乘矩阵)。
\begin{minted}{c++}
// U216697 线段树区间历史版本和 
// https://www.luogu.com.cn/problem/U216697
#include <bits/stdc++.h>

using namespace std;
using ll = long long;
const int  N = 100008;
int a[N], n, m;
struct Mat{
    static const int M = 3;
    ll m[M][M];
    Mat() { memset(m ,0 ,sizeof(m)); }
    Mat(ll k){
        memset(m ,0 ,sizeof(m)); 
        m[0][0] = m[1][1] = m[2][2]  = 1;
        m[0][1] = m[0][2] = k;
        m[1][2] = 1;
    }
    void eye() { for(int i = 0; i < M; i++) for(int j = 0; j < M; j++) m[i][j] = (ll)(i == j); }
    Mat operator * (const Mat& B) const{
        const Mat &A = *this;
        Mat ret;
        for(int k = 0; k < M; k++){
            for(int i = 0; i < M; i++)
                for(int j = 0; j < M; j++){
                    ret.m[i][j] = ret.m[i][j] + A.m[i][k] * B.m[k][j];
            }
        }
        return ret;
    }
    Mat operator + (const Mat& B) const{
        const Mat &A = *this;
        Mat ret;
        for(int i = 0; i < M; i++)
            for(int j = 0; j < M; j++)
                ret.m[i][j] = A.m[i][j] + B.m[i][j];
        return ret;
    }
};

struct Node {
    Mat v,tag;
} t[N << 2];
void pushup(int u) {
    t[u].v = t[u<<1].v + t[u<<1|1].v;
}
void build(int u, int l, int r) {
    t[u].tag.eye();
    if (l == r) {
        t[u].v.m[0][0] = 1;
        t[u].v.m[0][1] = a[l];
        t[u].v.m[0][2] = a[l];
        return;
    }
    int mid = (l + r) >> 1;
    build(u<<1, l, mid);
    build(u<<1|1, mid + 1, r);
    pushup(u);
}
void puttag(int u, Mat v) {
    t[u].v = t[u].v * v;
    t[u].tag = t[u].tag * v;
}
void pushdown(int u) {
    puttag(u<<1, t[u].tag);
    puttag(u<<1|1, t[u].tag);
    t[u].tag.eye();
}
void update(int u, int l, int r, int L, int R, ll v) {
    if (l >= L && r <= R) {
        puttag(u, Mat(v));
        return;
    }
    pushdown(u);
    int mid = (l + r) >> 1;
    if (L <= mid) update(u<<1, l, mid, L, R, v);
    if (R > mid)  update(u<<1|1, mid + 1, r, L, R, v);
    pushup(u);
}

Mat query(int u, int l, int r, int L, int R) {
    if (l >= L && r <= R) return t[u].v;
    
    pushdown(u);
    int mid = l + r >> 1;
    Mat ret;
    if (L <= mid) ret = ret + query(u<<1, l, mid, L, R);
    if (R > mid)  ret = ret + query(u<<1|1, mid + 1, r, L, R);
    return ret;
}

int main() {

    scanf("%d%d", &n, &m);
    for (int i = 1; i <= n; ++i) scanf("%d", &a[i]);
    build(1, 1, n);
    int op, x, y, l, r;
    for (int i = 1; i <= m; ++i) {
        scanf("%d", &op);
        if (op == 1) {
            scanf("%d%d%d", &l, &r, &x);
            update(1, 1, n, l, r, x);
            if(l - 1 >= 1)update(1, 1, n, 1, l - 1, 0);
            if(r + 1 <= n)update(1, 1, n, r + 1, n, 0);
        }
        else {
            scanf("%d%d", &x, &y);
            printf("%lld\n", query(1, 1, n, x, y).m[0][2]);
            update(1, 1, n, 1, n, 0);
        }
    }
    return 0;
}
\end{minted}
\subsection{不降子序列}
\par \noindent 线段树维护以某点为开头的最长不下降子序列,使用下面calc函数能够计算线段树\textbf{u维护的区间中},以$x$ 为开头最长不下降子序列的个数。calc需要维护区间最值。

\begin{minted}{c++}
template<typename T>
int calc(int u,T x)
{
    if(t[u].l==t[u].r) 
        return t[u].v>x?1:0;
    if(t[u<<1].v<=x) 
        return calc(u<<1|1,x);
    return t[u].cnt-t[u<<1].cnt+calc(u<<1,x);
}
\end{minted}

\subsection{维护矩阵}
\begin{tcolorbox}
\par \noindent 维护三个长度为 $n$ 的序列 $A, B, C$ ,支持以下 7 种操作:$n, m \leq 2.5 \times 10^5, 0 \leq A_{i}, B_{i}, C_{i}<998244353$
\begin{enumerate}
\item $l, r$ : 对 $[l, r], A_{i} \leftarrow A_{i}+B_{i}$
\item $l, r$ : 对 $[l, r],  B_{i} \leftarrow B_{i}+C_{i}$
\item $l, r$ : 对 $[l, r], C_{i} \leftarrow C_{i}+A_{i}$ 
\item $l, r,v$ : 对 $[l, r],  A_{i} \leftarrow A_{i}+v$
\item $l, r,v$ : 对 $[l, r],   B_{i} \leftarrow B_{i} \cdot v$ 
\item $l, r,v$ : 对 $[l, r],   C_{i} \leftarrow v$
\item $l,r$ : 求 $\sum_{i=l}^{r} A_{i}, \sum_{i=l}^{r} B_{i}, \sum_{i=l}^{r} C_{i}$ 在模 998244353 意义下。 
\end{enumerate}
\end{tcolorbox}
\begin{minted}{c++}
#include <bits/stdc++.h>
using namespace std;
typedef long long ll;
const int N = 250010;

const ll mod = 998244353;
struct dat {
    int n, m;
    ll a[5][5];
    dat() {}
    dat(int _n, int _m) {
        n = _n;
        m = _m;
        memset(a, 0, sizeof a);
    }
    friend dat operator +(const dat &X, const dat &Y) {
        //assert(X.n==Y.n);
        //assert(X.m==Y.m);
        dat ret(X.n, X.m);

        for (int i = 1; i <= X.n; i++)
            for (int j = 1; j <= X.m; j++)
                ret.a[i][j] = (X.a[i][j] + Y.a[i][j]) % mod;

        return ret;
    }
    friend dat operator *(const dat &X, const dat &Y) {
        dat ret(X.n, Y.m);
        assert(X.m == Y.n);
        for (int i = 1; i <= X.n; i++)
            for (int j = 1; j <= Y.m; j++)
                for (int k = 1; k <= X.m; k++)
                    ret.a[i][j] = (ret.a[i][j] + X.a[i][k] * Y.a[k][j]) % mod;
        return ret;
    }
};
struct node {
    int l, r;
    dat v, tag;
} t[N << 2];

int n, m;
dat num[N];
dat base1, base2, base3, base4, base5, base6;
dat base;
void build(int u, int l, int r) {
    t[u].l = l, t[u].r = r;
    if (l == r) {
        t[u].v = num[l];
        t[u].tag = base;
        return;
    }
    int mid = l + r >> 1;
    build(u << 1, l, mid), build(u << 1 | 1, mid + 1, r);
    t[u].v = t[u << 1].v + t[u << 1 | 1].v;
    t[u].tag = base;
}
void put(int u, dat &v) {
    t[u].v = t[u].v * v;
    t[u].tag = t[u].tag * v;
}
void pushdown(int u) {
    put(u << 1, t[u].tag);
    put(u << 1 | 1, t[u].tag);
    t[u].tag = base;
}
void modify(int u, int l, int r, dat v) {
    if (l <= t[u].l && t[u].r <= r) {
        t[u].v = t[u].v * v;
        t[u].tag = t[u].tag * v;
        return;
    }
    pushdown(u);
    int mid = t[u].l + t[u].r >> 1;
    if (l <= mid) modify(u << 1, l, r, v);
    if (r > mid) modify(u << 1 | 1, l, r, v);
    t[u].v = t[u << 1].v + t[u << 1 | 1].v;
}
dat query(int u, int l, int r) {
    if (l <= t[u].l && t[u].r <= r)
        return t[u].v;
    pushdown(u);
    int mid = t[u].l + t[u].r >> 1;
    dat v(1, 4);
    if (l <= mid) v = v + query(u << 1, l, r);
    if (r > mid) v = v + query(u << 1 | 1, l, r);
    return v;
}

void init() {
    base = dat(4, 4);
    for (int i = 1; i <= 4; i++)
        base.a[i][i] = 1;

    /*
    base1:| base2:| base3:
    1 0 0 | 1 0 0 | 1 0 1
    1 1 0 | 0 1 0 | 0 1 0
    0 0 1 | 0 1 1 | 0 0 1
    */
    base1 = base;
    base1.a[2][1] = 1;
    base2 = base;
    base2.a[3][2] = 1;
    base3 = base;
    base3.a[1][3] = 1;
    /*
    base4:  | base5:  | base6:
    1 0 0 0 | 1 0 0 0 | 1 0 0 0
    0 1 0 0 | 0 v 0 0 | 0 1 0 0
    0 0 1 0 | 0 0 1 0 | 0 0 0 0
    v 0 0 1 | 0 0 0 1 | 0 0 v 1
    */
    base4 = base; // [4][1]=v
    base5 = base; // [2][2]=v;
    // [4][3]=v;
    base6 = base;
    base6.a[3][3] = 0;
}


int main() {
    init();
    cin >> n;
    for (int i = 1; i <= n; i++) {
        num[i] = dat(1, 4);
        cin >> num[i].a[1][1] >> num[i].a[1][2] >> num[i].a[1][3];
        num[i].a[1][4] = 1;
    }
    build(1, 1, n);
    cin >> m;

    while (m--) {
        int op, l, r, v;
        cin >> op >> l >> r;
        if (op == 1)
            modify(1, l, r, base1);
        else if (op == 2)
            modify(1, l, r, base2);
        else if (op == 3)
            modify(1, l, r, base3);
        else if (op == 4) {
            cin >> v;
            base4.a[4][1] = v;
            modify(1, l, r, base4);
        } else if (op == 5) {
            cin >> v;
            base5.a[2][2] = v;
            modify(1, l, r, base5);
        } else if (op == 6) {
            cin >> v;
            base6.a[4][3] = v;
            modify(1, l, r, base6);
        } else {
            dat ret = query(1, l, r);
            cout << ret.a[1][1] << ' ' << ret.a[1][2] << ' ' << ret.a[1][3] << '\n';
        }
    }

    return 0;
}
\end{minted}
\subsection{区间 GCD}

\begin{tcolorbox}
给定一个长度为 N 的数列 A,以及 M 条指令,每条指令可能是以下两种之一:
\begin{itemize}
\item C l r d,表示把 $A[l],A[l+1],…,A[r]$ 都加上 $d$。
\item Q l r,表示询问 $A[l],A[l+1],…,A[r]$ 的最大公约数。
\end{itemize}
\end{tcolorbox}

$$
\operatorname{gcd}\left(a_1, a_2, \cdots, a_n\right)=\operatorname{gcd}\left(a_1, a_2-a_1, a_3-a_2, \cdots, a_n-a_{n-1}\right)
$$

\par \noindent 根据性质用 $b[i] = a[i] - a[i-1]$数组,表示原序列的差分序列,用线段数维护 $b[i]$ 的区间的最大公约数
\begin{itemize}
\item 询问 C l r d  等价于
\begin{enumerate}
\item $b[l]$ 加上 $d$、$b[r+1]$ 减去 $d$ ,只需两次线段数的单点修改即可
\item 对于原序列中 $a[i]$ 的值,需要支持区间加、点单查询,只需要用树状数组通过维护差分序列即可。
\end{enumerate}
\item 询问 Q l r 等于求出 $\operatorname{gcd}(a[l]$, query $(1, l+1, r))$
\end{itemize}
\begin{minted}{c++}
#include<bits/stdc++.h>

using namespace std;
using ll = long long;
const int N = 500010;

int n, m;
ll a[N];
struct Node {
    int l, r;
    ll v;
}t[N<<2];
void pushup(int u) {
    t[u].v = __gcd(t[u << 1].v, t[u << 1 | 1].v);
}
void build(int u, int l, int r) {
    t[u]={l, r};
    if (l == r) {
        t[u].v = a[l] - a[l - 1];
        return;
    }
    int mid = l + r >> 1;
    build(u << 1, l, mid);
    build(u << 1 | 1, mid + 1, r);
    pushup(u);
}
void modify(int u, int pos, ll x) {
    if(t[u].l == t[u].r) {
        t[u].v += x;
        return;
    }
    int mid = t[u].l + t[u].r >> 1;
    if(pos <= mid) modify(u << 1, pos, x);
    else modify(u << 1 | 1, pos, x);
    pushup(u);
}
ll query(int u, int l, int r) {
    if(l <= t[u].l && t[u].r <=r) {
        return t[u].v;
    }
    int mid = t[u].l + t[u].r >> 1;
    ll v = 0;
    if(l <= mid) v = __gcd(v, query(u << 1, l, r));
    if(r > mid) v = __gcd(v, query(u << 1 | 1, l, r));
    return v;
}

ll bit[N];
void add(int k, ll x) {
    for(; k <= n; k += k&-k) bit[k] += x;
}
ll sum(int k) {
    ll v = 0;
    for(; k; k -= k&-k) v += bit[k];
    return v;
}
int main() {

    cin >> n >> m;
    for (int i = 1; i <= n; i++) cin >> a[i];
    // 差分数组
    build(1, 1, n);

    // 原数组 -> 树状数组差分维护
    for (int i = 1; i <= n; i ++) add(i, a[i] - a[i-1]);

    while (m--) {
        char op;
        int l, r; 
        cin >> op >> l >> r;
        if (op =='Q') {
            ll x = sum(l);
            ll y = query(1, l + 1, r);
            cout << abs(__gcd(x, y)) << '\n';
        }
        else {
            ll val; cin >> val;
            // 原数组 区间加
            add(l, val);
            if(r + 1 <= n) add(r + 1, -val);
            // 差分数组
            modify(1, l, val);
            if (r + 1 <= n) modify(1, r + 1, -val);
        }
    }

    return 0;
}
\end{minted}


有时候维护集合的 gcd 并支持:
\begin{enumerate}
\item 合并两个 gcd 集合
\item 对某些集合的所有数加上一个值
\end{enumerate}
$$
\operatorname{gcd}\left(a_1, a_2, \cdots, a_n\right)=\operatorname{gcd}\left(a_1, a_2-a_1, a_3-a_1, \cdots, a_n-a_{1}\right)
$$

\par \noindent 对于每一个集合维护两个数 
$$
f = a_1,g = \gcd(a_2-a_1, a_3-a_1, \cdots, a_n-a_{1})
$$
\par \noindent 对一个集合的所有数加上一个值操作 $f$ 即可。下面考虑如何合并两个集合:
~\\
\par \noindent 两个 gcd 合并时 需要将 $f_1$ 和 $f_2$ 合并 $g_1$ 和 $g_2$ 合并,令 $f_1$ 和 $f_2$ 合并成 $f_1$ (意味着对于两个序列合并,维护第一个序列的$a_1$)考虑差分的性质,对第二个集合来说等价于所有数减去 $f_1$,于是合并后集合应该有:
$$
f = f_1, g = \gcd(g_1, g_2,f_2 -f_1)
$$
\subsection{区间NAND}

\par \noindent 区间与非\textbf{没有结合律},这样的信息线段树不能直接维护,不过位运算具有独立性,我们可以一位一位去考虑。
~\\
\par \noindent 考虑用线段树每个节点维护$\text{L}[0/1],\text{R}[0/1]$
\begin{itemize}
\item $\text{L}[0]$表示刚开是$0$,然后从左向右经过此区间是最终的数(此节点维护的区间)
\item $\text{L}[1]$表示刚开是$1$,然后从左向右经过此区间是最终的数
\item $\text{R}[0]$表示刚开是$0$,然后从右向左经过此区间是最终的数
\item $\text{R}[1]$表示刚开是$1$,然后从右向左经过此区间是最终的数
\end{itemize}
\par \noindent 然后只需要维护32棵线段树(按位),就可以区间询问了。而且区间是从左向右,有的区间是从右向左的答案也不一样!

\begin{minted}{c++}
struct Segment
{
    struct node
    {
        int l,r;
        bool L[2],R[2];
    }tree[N<<2];
    void pushup(int u)
    {
        tree[u].L[0]=tree[u<<1|1].L[tree[u<<1].L[0]];
        tree[u].L[1]=tree[u<<1|1].L[tree[u<<1].L[1]];
        tree[u].R[0]=tree[u<<1].R[tree[u<<1|1].R[0]];
        tree[u].R[1]=tree[u<<1].R[tree[u<<1|1].R[1]];
    }
    void build(int u,int l,int r,int k)
    {
        tree[u].l=l,tree[u].r=r;
        if(l==r) 
        {
            tree[u].L[0]=tree[u].R[0]=1;
            tree[u].L[1]=tree[u].R[1]=!(a[id[l]]>>k&1);
            return;
        }
        int mid=l+r>>1;
        build(u<<1,l,mid,k);build(u<<1|1,mid+1,r,k);
        pushup(u);
    }
    void modify(int u,int pos,bool x)
    {
        if(tree[u].l==tree[u].r)
        {
            tree[u].L[0]=tree[u].R[0]=1;
            tree[u].L[1]=tree[u].R[1]=(!x);
            return;
        }
        int mid=tree[u].l+tree[u].r>>1;
        if(pos<=mid) 
            modify(u<<1,pos,x);
        else
            modify(u<<1|1,pos,x);
        pushup(u);
    }
    bool queryL(int u,int l,int r,bool c)
    {
        if(l<=tree[u].l&&tree[u].r<=r) return tree[u].L[c];
        int mid=tree[u].l+tree[u].r>>1;
        if(r<=mid)
            return queryL(u<<1,l,r,c);
        else if(l>mid) 
            return queryL(u<<1|1,l,r,c);
        else 
            return queryL(u<<1|1,l,r,queryL(u<<1,l,r,c));
        
    }
    bool queryR(int u,int l,int r,bool c)
    {
        if(l<=tree[u].l&&tree[u].r<=r) return tree[u].R[c];
        int mid=tree[u].l+tree[u].r>>1;
        if(r<=mid)
            return queryR(u<<1,l,r,c);
        else if(l>mid) 
            return queryR(u<<1|1,l,r,c);
        else 
            return queryR(u<<1,l,r,queryR(u<<1|1,l,r,c));
    }
}T[33];
\end{minted}
\subsection{主席树}
\par \noindent 可持久化线段树的主要思想:保存每次插入操作的历史版本,实际上是在动态开点线段树的基础上,通过复用某些未修改的节点,创建 $n$ 棵线段树。
~\\
\par \noindent 每进行一次修改时,产生新的节点数 = 树的高度,是$O(n \log n)$ 级别的。

\begin{minted}{c++}
struct node
{
    int l,r;
    ll v;
}t[N*40];
int rt[N],cnt;
void ins(int &u,int o,int l,int r,int pos)
{
    u=++cnt;
    t[u]=t[o];
    t[u].v+=pos;
    if(l==r) return;
    int mid=l+r>>1;
    if(pos<=mid) 
        ins(t[u].l,t[o].l,l,mid,pos);
    else
        ins(t[u].r,t[o].r,mid+1,r,pos);
}
ll query(int u,int l,int r,int L,int R)
{
    if(!u) return 0ll;
    if(L<=l&&r<=R) return t[u].v;
    int mid=l+r>>1;
    ll v=0;
    if(L<=mid)   
        v+=query(t[u].l,l,mid,L,R);
    if(R>mid)
        v+=query(t[u].r,mid+1,r,L,R);
    return v;
}
\end{minted}

\subsection{动态开点权值线段树}
\par \noindent 有的时候,线段树需要维护的区间很大很大,但是实际用到的节点很少;那么干脆就不要开这么多的节点,用到的时候再向内存要。

\begin{minted}{c++}
namespace DynamicSegTree {
    struct Node {
        int l, r, v;
    }t[maxn*40];
    int cnt;
    // 修改操作,若 u 不存在,则首先动态开点;否则直接更新
    void update(int &u, int l, int r, int pos, int val) {
        if(!u) u = ++cnt;
        // ....
    }
    // 查询操作区间 [L, R] ,若 u 不存在直接返回空值(可能是 0 或者反向最值)
    int query(int u, int l, int r, int L, int R) {
        if(!u) 
            return 0;
        // .....
    }

}using namespace DynamicSegTree;
\end{minted}
\subsection{线段树合并}
\begin{minted}{c++}
int merge(int x,int y,int l,int r)
{
    if(!x||!y) return x+y;
    int mid=l+r>>1;
    if(l==r)
    {
        t[x].val+=t[y].val;
        return x;
    }
    t[x].l=merge(t[x].l,t[y].l,l,mid);
    t[x].r=merge(t[x].r,t[y].r,mid+1,r);
    t[x].val=max(t[t[x].l].val,t[t[x].r].val);
    return x;
}
\end{minted}

\subsection{Segment Tree Beats!}
\par \noindent 区间最值操作往往采用以下办法 

\par \noindent 线段树维护:
\begin{itemize}
\item 区间最大值$\text{mx}$  
\item 区间严格次大值$\text {smx}$  
\item 区间和$\text{sum}$ 
\item 区间最大值个数$\text{cnt}$  
\item 区间最值懒标记$\text{lazy}$
\end{itemize}
\par \noindent 实现区间最小值操作,考虑u节点维护的区间,进行如下处理
\begin{itemize}
\item 当$\text{mx}\leq x$,显然这次修改不会对这个节点维护的区间产生影响,直接退出。
\item 当$\text{smx}<x<\text{mx}$,显然这次修改只会影响到这个区间所有的最大值,由此直接根据最大值个数更新区间和并且更新区间最大值并打上懒标记然后退出即可。
\item 当$\text{x}\leq \text{smx}$,无法直接更新于是递归左右子树。
\end{itemize}

\begin{minted}{c++}
struct Node {
    int l, r; // 最大值 次大值 区间和
    ll maxv, secmaxv, sum;
    int cnt;
    ll tag;
} t[N << 2];
int n, m;
ll a[N];
void pushup(int u) {
    int x = u << 1, y = u << 1 | 1;
    t[u].sum = t[x].sum + t[y].sum;
    // 保证 t[x].maxv > t[u].maxv
    if (t[x].maxv < t[y].maxv) swap(x, y);
    
    if (t[x].maxv != t[y].maxv) {
        t[u].maxv = t[x].maxv;
        t[u].secmaxv = max(t[x].secmaxv, t[y].maxv);
        t[u].cnt = t[x].cnt;
    } else {
        t[u].maxv = t[x].maxv;
        t[u].secmaxv = max(t[y].secmaxv, t[y].secmaxv);
        t[u].cnt = t[x].cnt + t[y].cnt;
    }
}
void puttag(int u, ll x) {
    if (t[u].maxv <= x) return;
    t[u].sum += (x - t[u].maxv) * t[u].cnt;
    t[u].maxv = t[u].tag = x;
}
void pushdown(int u) {
    if (t[u].tag == -1) return;
    puttag(u << 1, t[u].tag), puttag(u << 1 | 1, t[u].tag);
    t[u].tag = -1;
}
void build(int u, int l, int r) {
    t[u] = {l, r, 0, -1, 0, 0, -1};
    if (l == r) {
        t[u].maxv = t[u].sum = a[l];
        t[u].cnt = 1;
        return;
    }
    int mid = l + r >> 1;
    build(u << 1, l, mid), build(u << 1 | 1, mid + 1, r);
    pushup(u);
}
void modify(int u, int l, int r, ll x) { // 区间[l, r] a[i] = min(a[i], x)
    if (x >= t[u].maxv) return;
    // 当修改只会影响到区间最大值,并且不影响次大值时修改 意味着 secmaxv < x < maxv
    // 此时需要把区间最大值修改成 x
    if (t[u].l >= l && t[u].r <= r && t[u].secmaxv < x) {
        puttag(u, x);
        return;
    }
    pushdown(u);
    int mid = t[u].l + t[u].r >> 1;
    if (l <= mid) modify(u << 1, l, r, x);
    if (r > mid) modify(u << 1 | 1, l, r, x);
    pushup(u);
}
ll qmax(int u, int l, int r) {
    if (t[u].l >= l && t[u].r <= r) return t[u].maxv;
    int mid = t[u].r + t[u].l >> 1;
    pushdown(u);
    ll v = -1;
    if (l <= mid) v = max(v, qmax(u << 1, l, r));
    if (r > mid) v = max(v, qmax(u << 1 | 1, l, r));
    pushup(u);
    return v;
}
ll qsum(int u, int l, int r) {
    if (t[u].l >= l && t[u].r <= r)
        return t[u].sum;
    int mid = t[u].r + t[u].l >> 1;
    pushdown(u);
    ll v = 0;
    if (l <= mid) v += qsum(u << 1, l, r);
    if (r > mid) v += qsum(u << 1 | 1, l, r);
    pushup(u);
    return v;
}
\end{minted}

\begin{tcolorbox}
\par \noindent 维护一个长度为 $n$ 的序列,支持 $m$ 次操作,操作包括区间按位或一个数,区间按位与一个数,以及查询区间最大值。
\end{tcolorbox}

\par \noindent 线段树每个节点上维护区间与、区间或和区间最大值。
如果一次操作对区间与的影响和对区间或的影响相同,那么就说明对这整个区间的影响都是相同的,就是加上或减去同一个值,直接打标记即可,否则递归下去处理。
