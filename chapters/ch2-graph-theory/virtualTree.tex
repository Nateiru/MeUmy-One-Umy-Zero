\section{虚树}

\par \noindent 给定一棵树,有多次询问,每组询问对树上的一些关键点,询问它们的某些信息,满足所有询问中的关键点数量\textbf{总和}与树的大小同阶。此时可以对原树建立一棵只包含关键点和 LCA 的虚树,可以将问题的复杂度压缩到 $O(Q \log_2N + f(Q))$ 的等级。
~\\
\par \noindent \textbf{关键性质:}在一棵有根树中,任选 $k$ 个不重复的点,这  $k$ 个点两两之间的不同 LCA 数量不超过  $k-1$  个(即  $k$ 个点两两之间的 LCA 数量不超过自身的阶数)。
\begin{minted}{c++}
namespace VirtualTree {
    
    vector<int> VT[maxn];
    
    inline bool cmp(const int &i, const int &j) {
        return dfn[i] < dfn[j];
    }
    // 构建虚树
    void build_virtual_tree(int node[], int k) {
        static int stk[maxn];
        // k 个关键节点
        sort(node, node + k, cmp), top = 0;
        // stk[top = 1]  = 1; // 根节点入栈
        for (int i = 0; i < k; i++) {   
            if (i == 0)
                assert(node[i] == 1); // 根节点是关键节点
            if (top <= 1) {
                stk[++top] = node[i];
                continue;
            }
            int u = node[i], p = lca(u, stk[top]);
            if (p == stk[top]) {
                stk[++top] = u;
                continue;
            }
            while (top > 1 && dfn[p] <= dfn[stk[top-1]])
                VT[stk[top-1]].emplace_back(stk[top]), top--;
            if (p != stk[top])
                VT[p].emplace_back(stk[top]), stk[top] = p;
            stk[++top] = u;
        }
        for (int i = 1; i < top; i++)
            VT[stk[i]].emplace_back(stk[i+1]);
    }

    void calc_vt_depth(int x, int p) {
        fa[x] = p, vdepth[x] = vdepth[p] + 1;
        for (int i : VT[x])
            calc_vt_depth(i, x);
    }
    // void clear()  构建的时候清空 或者dfs过程遍历完就清空
}using namespace VirtualTree;

// 处理路径 (u, v) 上的更新和询问 
ll a[maxn], w[maxn];
// 更新
while (u != v) {
    if (rdp[u] < rdp[v])
        swap(u, v);
    w[u] += x, a[u] += x, u = fa[u];
}
a[u] += x;
// 询问
ll ans = 0;
while (u != v)  {
    if (rdp[u] < rdp[v])
        swap(u, v);
    if (depth[u] - depth[fa[u]] - 1)
        ans += w[u] * (depth[u] - depth[fa[u]] - 1);    // 用 depth 统计 
    ans += a[u];
}
ans += a[u];
\end{minted}