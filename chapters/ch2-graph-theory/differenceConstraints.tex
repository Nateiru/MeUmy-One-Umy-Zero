\section{差分约束}

\par \noindent 求解差分约束系统,有 $m$ 条约束条件,每条都为形如 $x_a-x_b\geq c_k$,$x_a-x_b\leq c_k$ 或 $x_a=x_b$ 的形式,判断该差分约束系统有没有解。
~\\
\par \noindent 
\par \noindent 跑判断负环,如果不存在负环,输出 `Yes`,否则输出 `No`。
~\\
\par \noindent 如果${x_1,x_2,...,x_n}$是该差分约束系统的一组解,那么对于任意常数 $d$, ${x_1+d,x_2+d,...,x_n+d}$显然也是该差分约束系统的一组解,因为这样做差后 $d$ 会被消掉。
~\\
\par \noindent \textbf{步骤:} 

\begin{enumerate}
\item 先将每个不等式$x_a - x_b \leq c$,转换成一条从 $x_b$ 走到 $x_a$,长度为 $c$ 的边。
\item 找到一个超级源点,使得该源点一定可以走到所以边
\item 从源点求一遍单源最短路
\end{enumerate}

\par \noindent \textbf{结果1:}如果存在负环,则原不等式组一定无解
\par \noindent \textbf{结果2:}如果没有负环,则 $dist[i]$ 就是原不等式组的一个可行解
~\\
\par \noindent \textbf{最大值和最小值结论:}

\begin{itemize}
\item 如果求的是最小值,则应该求最长路

\item 如果求的是最大值,则应该求最短路
\end{itemize}

\par \noindent 比如求\textbf{最小解},只需要求\textbf{最长路}就行了。最长路满足三角形不等式$dist[u]≥dist[v]+w_{v,u}$