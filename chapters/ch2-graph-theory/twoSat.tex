\section{2-SAT 问题}
\begin{tcolorbox}
\par \noindent \textbf{定义}:简单的说就是给出n个集合,每个集合有两个元素,已知若干个$<a,b>$,表示 $a$ 与 $b$ 矛盾(其中 $a$ 与 $b$ 属于不同的集合)。然后从每个集合选择一个元素,判断能否一共选n个两两不矛盾的元素。
\end{tcolorbox}
\par \noindent \textbf{原理}: 首先建图: 假设两个集合 $\left\{a_1, b_1\right\}$ 和 $\left\{a_2, b_2\right\}$, 如果 $a_1, b_2$ 冲突, 那么连有向边 $\left(a_1, b_1\right)$ 和 $\left(b_2, a_2\right)$, 然后跑一遍 Tarjan 有向图缩点, 判断是否有一个集合的两个元素同时在同一个 SCC 中, 如果有则无 解, 否则有解。
~\\
\par \noindent \textbf{输出方案}: Tarjan 算法求强连通分量时用了栈, 求得各点所在的 SCC 编号相当于反拓扑序。对于任 意集合可以表示为 $\{x, \neg x\}$; 如果变量 $\neg x$ 的拓扑序在 $x$ 之后(即 $\operatorname{topo}(\neg x) \geq \operatorname{topo}(x)$ ), 则取 $x$ 为真。
\begin{minted}{c++}
namespace TwoSAT {
    #define clear(x) memset(x, 0, sizeof(x))
    int head[maxn], dfn[maxn], low[maxn], c[maxn], stk[maxn];
    int top = 0, scccnt = 0, order = 1, cnt = 1;
    bool instack[maxn];
    struct Edge {
        int u, v, next;
        Edge(int u = 0, int v = 0, int next = 0): u(u), v(v), next(next) {}
    } e[maxm];
    void add_edge(int u, int v) {
        e[cnt] = Edge(u, v, head[u]);
        head[u] = cnt++;
    }
    void init() {
        clear(dfn), clear(low), clear(c), clear(instack), clear(head);
        scccnt = 0, order = 1, cnt = 1, top = 0;
    }
    void tarjan(int x) {
        dfn[x] = low[x] = order++;
        stk[++top] = x, instack[x] = 1;
        for (int i = head[x]; i; i = e[i].next) {
            int y = e[i].v;
            if (!dfn[y])
                tarjan(y), low[x] = min(low[x], low[y]);
            else if (instack[y])
                low[x] = min(low[x], dfn[y]);
        }
        if (dfn[x] == low[x]) {
            int tmp;
            do {
                tmp = stk[top--];
                c[tmp] = scccnt, instack[tmp] = 0;
            } while (tmp != x);
            scccnt++;
        }
    }
    void shrink(int n) {
        for (int x = 0; x <= n; x++)
            if (!dfn[x])
                tarjan(x);
    }
    bool solve(int n) {
        shrink(n);
        for (int i = 0; i <= n; i += 2)
            if (c[i] == c[i+1])
                return false;
        return true;
    }
}using namespace TwoSAT;
\end{minted}