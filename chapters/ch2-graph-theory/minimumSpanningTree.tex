\section{最小生成树}

\subsection{Kruskal 算法}
\begin{minted}{c++}
// n 个点 m 条边的最小生成树
int p[maxn], n, m;
int find(int x) {
    return x == p[x] ? x : p[x] = find(p[x]);
}
int main() {
    // ...
    for (int i = 1; i <= n; i++) p[i] = i;
    sort(e + 1, e + 1 + m);
    int cnt = 0, ans = 0;
    for (int i = 1; i <= m; i++) {
        int px = find(e[i].u), py = find(e[i].v);
        if (px != py) {
            p[px] = py;
            ans += e[i].w;
            cnt++;
        }
    }
    if (cnt != n - 1) // 
        printf("No solution\n");
    else
        printf("%d", ans);
    return 0;
}
\end{minted}

\subsection{堆优化的 Prim 算法}
\begin{minted}{c++}
#include<bits/stdc++.h>

using namespace std;
const int maxn=200010;
struct Edge {
    int u, v, w, next;
} e[maxn << 1];
struct Node {
    int v, w;
    bool operator < (const Node& o) const {
        return w > o.w;
    }
};
priority_queue<Node> q;
int n, m, head[maxn], cnt = 1;
bool vis[maxn];
void add_edge(int u, int v, int w) {
    e[cnt] = (Edge) { u, v, w, head[u] };
    head[u] = cnt++;
}

int main() {
    scanf("%d%d", &n, &m);
    for(int i = 0; i < m; i++) {    
        int a, b, w;
        scanf("%d%d%d", &a, &b, &w);
        add_edge(a, b, w);
        add_edge(b, a, w);
    }
    // ...
    vis[0] = 1;
    q.push((Node) {1, 0}); // 1 号节点的边权是0
    long long ans = 0;
    int cnt = 0;
    while (!q.empty() && cnt <= n) {
        Node t = q.top();
        q.pop();
        if (vis[t.v]) continue; // 已经在最小生成树中
        vis[t.v] = 1;
        ans += t.w;
        cnt++;
        for (int i = head[t.v]; i; i = e[i].next) {
            if (!vis[e[i].v])
                q.push((Node) { e[i].v, e[i].w });
        }
    } 
    if (cnt != n) 
        printf("No solution\n");
    else 
        printf("%lld", ans);
    
}
\end{minted}

\subsection{最小瓶颈路}
\par \noindent 给定一个加权无向图,并给定无向图中两个结点 $u$ 和 $v$,求 $u$ 到 $v$ 的一条路径,使得路径上边的最大权值最小。
~\\
\par \noindent 可以证明这个“最大权值最小”的边一定在最小生成树上。对于询问两个点 $u, v$ 的最小瓶颈路的问题,我们对求完的最小生成树求一遍最近公共祖先 LCA,两者到达 LCA 的路径中的最大边就是最小瓶颈路了。
\begin{minted}{c++}
// 倍增求LCA过程 同时预处理 ancestor 和 cost 数组
int query(int x, int y) {
  int d = 0;
  if (dep[x] < dep[y]) swap(x, y);
  for (d = 0; (1 << (d + 1)) <= dep[x]; d++);
  
  int ans = -1;
  for (int i = d; i >= 0; i--)
    if (dep[x] - (1 << i) >= dep[y]) {
      ans = max(ans, cost[x][i]);
      x = ancestor[x][i];
    }
  if (x == y)
    return ans;
  for (int i = d; i >= 0; i--)
    if (ancestor[x][i] > 0 && ancestor[x][i] != ancestor[y][i]) {
      ans = max(ans, max(cost[x][i], cost[y][i]));
      x = ancestor[x][i], y = ancestor[y][i];
    }
  ans = max(ans, max(cost[x][0], cost[y][0]));
  return ans;
}
\end{minted}
\subsection{最小直径生成树}
\textbf{定义:} 在无向图的所有生成树中,直径长度最小的一棵生成树。
\begin{minted}{c++}
bool cmp(int a, int b) {
    return val[a] < val[b];
}
void floyd() {
    for (int k = 1; k <= n; k ++)
        for (int i = 1; i <= n; i ++)
            for (int j = 1; j <= n; j ++)
                d[i][j] = min(d[i][j], d[i][k] + d[k][j]);
}
int solve() {
    floyd();
    for (int i = 1; i <= n; i ++) {
        for (int j = 1; j <= n; j ++) 
            rk[i][j] = j, val[j] = d[i][j];
        sort(rk[i] + 1, rk[i] + 1 + n, cmp);
    }
    int ans = INF;
    // 图的绝对中心可能在结点上
    for (int i = 1; i <= n; i ++)
        ans = min(ans, d[i][rk[i][n]] * 2);
    // 图的绝对中心可能在边上
    for (int i = 1; i <= m; i ++) {
        int u = a[i].u, v = a[i].v, w = a[i].w;
        for (int p = n, i = n - 1; i >= 1; i --)
            if (d[v][rk[u][i]] > d[v][rk[u][p]])
                ans = min(ans, d[u][rk[u][i]] + d[v][rk[u][p]] + w),
                p = i;
    }
    return ans;
}
\end{minted}
\subsection{Kruskal 重构树}
\textbf{建树步骤:}

\begin{enumerate}
\item 将所有边按边权从小到大排序 
\item 顺序遍历每条边 $(u,v,w)$,若 $u,v$ 已经联通跳过,否则建立一个新点 $x$,让$x$ 作为$p_u$ 与 $p_v$ 的父亲(即连 $x\to p_u$ 和 $x\to p_v$ 的有向边),然后让 $p_u=p_v=x$。这个新点的点权是$w$。
   $O(m\log m)$
\end{enumerate}
\textbf{性质:}
\begin{enumerate}
\item 原图中的点在重构树中一定是叶子节点,其余节点都代表了一条边的边权。
\item 重构树中的点数是$2n-1$且以$2n-1$号点为根节点
\item 如果边权按照从小到大排序建立重构树,重构树的点权是一个大根堆,反之小根堆
\item 对于一个 $x$ 和一个值 $v$。从 $x$ 出发只经过 $≤v$的边能到达的点集 $= x$ 的祖先节点中深度最小的点权 $≤v$ 的点 $z$ 的子树中的原来的点集。
\item 原树两点间的的最大边权就是kruskal重构树上两点的LCA的权值。
\end{enumerate}
\begin{minted}{c++}
#include <bits/stdc++.h>
using namespace std;
const int N = 200005, M = 400005;
struct Edge {
    int u, v, w;
    bool operator<(const Edge &o)const {
        return w > o.w;
    }
} edge[M];
vector<int> G[N << 1];
int fa[N << 1];
int find(int x) {
    return x == fa[x] ? x : fa[x] = find(fa[x]);
}
int tot, val[N << 1];
void Merge(int u, int v, int w) {
    u = find(u), v = find(v);

    if (u == v) return;

    fa[u] = fa[v] = ++tot;
    val[tot] = w;
    G[tot].push_back(u);
    G[tot].push_back(v);
}
int n, m;
void build() {
    for (int i = 1; i <= 2 * n; i++) fa[i] = i;

    sort(edge + 1, edge + 1 + m);

    for (int i = 1; i <= m; i++) Merge(edge[i].u, edge[i].v, edge[i].w);
}
int main() {
    return 0;
}
\end{minted}