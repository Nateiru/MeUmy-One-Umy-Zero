\section{分治}
\subsection{三分法}
\noindent \par 三分查找是在二分查找分出了两个区间(左区间,右区间)的情况下,再对左区间或右区间进行一次二分,以快速确定最值。三分法要求序列是一个有凹凸性的函数。步骤如下:

\begin{itemize}
    \item 取得区间的中间值 $mid = \lfloor \frac{(left + right)}{2} \rfloor$
    \item 取右区间的中间值 $rmid = \lfloor \frac{(mid + right)}{2} \rfloor$
    \item 判断 $a[rmid]$ 和 $a[mid]$ 的关系,若 $a[mid]$ 比 $a[rmid]$ 更接近最值,舍弃右区间搜索左区间;否则搜索左区间。
\end{itemize}

\begin{minted}{c++}
int minimun_int(int L, int R) { //三分求 f 函数的最小值(定义域为整数)
    while (R > L) {
        int m1 = (2 * L + R) / 3;
        int m2 = (2 * R + L + 2) / 3;
        if (f(m1) < f(m2))  // f(m1) > f(m2) 求最大值
            R = m2 - 1;
        else
            L = m1 + 1;
    }
    return L; //f(L) 为最小值
}
double maximum_double(double L, double R) { //三分求 f 函数的最大值(定义域为实数)
    while (R - L > eps) { // for i in range(100):
        double m1 = (2 * L + R) / 3;
        double m2 = (2 * R + L) / 3;
        if (f(m1) > f(m2))  // f(m1) > f(m2) 求最小值
            R = m2;
        else
            L = m1;
    }
    return L; //f(L) 为最大值
}
\end{minted}

\subsection{归并排序}
\begin{minted}{c++}
int n, a[100010], tmp[100010];
void merge_sort(int a[], int l, int r) {
    if (l >= r) return;

    int mid = l + r >> 1;
    merge_sort(a, l, mid), merge_sort(a, mid + 1, r);
    int i = l, j = mid + 1, k = 0;

    while (i <= mid && j <= r) {
        if (a[i] <= a[j])
            tmp[k++] = a[i++];
        else
            tmp[k++] = a[j++];
    }

    while (i <= mid) tmp[k++] = a[i++];
    while (j <= r)   tmp[k++] = a[j++];

    for (i = l, k = 0; i <= r; i++, k++) a[i] = tmp[k];
}
\end{minted}

\subsection{平面最近点对}
\begin{minted}{c++}
struct Point {
    int x,y;
    bool type; // 两种类型的平面最近点对
    bool operator <(const Point &o) const {
        return x<o.x;
    }
}point[N],tmp[N];
int n;
double dis(Point a,Point b) {
    // if(a.type==b.type) return INF;
    double dx=a.x-b.x,dy=a.y-b.y;
    return sqrt(dx*dx+dy*dy);
}
double solve(int l,int r)
{
    if(l==r) return INF;
    int mid=l+r>>1;
    double flag=point[mid].x;
    // 分治计算出上述未被更新的 ans
    double ans=min(solve(l,mid),solve(mid+1,r));
    // 先将 points 中的 [l, mid] 和 [mid + 1, r] 两段进行按 y 轴坐标进行按序归并
    // 注意这里一定要归并,后面对于每个点我们才能快速找出对应的(至多) 6 个点,以保证总时间复杂度是 O(n log n)

    inplace_merge(point+l,point+mid+1,point+r+1,[](const Point&a,const Point&b){return a.y<b.y;});
    // 找到所有在 [mid_x - ans, mid_x + ans] 中的点,存入 tmp
    int k=0;
    for(int i=l;i<=r;i++)
        if(point[i].x>=flag-ans&&point[i].x<=flag+ans)
            tmp[k++]=point[i];
    // 下面第二层循环中,有 tmp[i].y - tmp[j].y <= ans 这个判断,才能保证我们对于每个点最多只考虑六个点
    for(int i=0;i<k;i++)
        for(int j=i-1;j>=0&&tmp[i].y-tmp[j].y<ans;j--)
            ans=min(ans,dis(tmp[i],tmp[j]));

    return ans;
}
\end{minted}


\subsection{CDQ 分治求三维偏序}
\begin{minted}{c++}
// update 和 query 是单点修改区间查询的数据结构
struct Node {
    int a, b, c;
    int cnt, ans;
    bool operator<(const Node &o)const  {
        return a < o.a || a == o.a && b < o.b || a == o.a && b == o.b && c < o.c;
    }
    bool operator==(const Node &o)const {
        return a == o.a && b == o.b && c == o.c;
    }
} q[N];
void cdq(int l, int r) {
    if (l >= r) return;

    int mid = l + r >> 1;
    cdq(l, mid), cdq(mid + 1, r);

    int i = l;
    for (int j = mid + 1; j <= r; j++) {
        while (i <= mid && q[i].b <= q[j].b)
            update(q[i].c, q[i].cnt), i++; 
        q[j].ans += query(q[j].c);
    }

    while (i > l) --i, update(q[i].c, -q[i].cnt);

    inplace_merge(q + l, q + mid + 1, q + r + 1, [](const Node & x, const Node & y) {
        return x.b < y.b;
    });
}
\end{minted}
\subsection{四维偏序}
\begin{tcolorbox}
\par \noindent 在一个三维空间当中,每次进行一个操作,添加一个点或者统计空间中的某一个\textbf{长方体}范围内的所有点
\end{tcolorbox}

\par \noindent 三维空间中我们用两个点即可确定一个长方体。
~\\
\par \noindent 首先效仿平面二维数点的方法,根据容斥原理可以把询问拆分成8个以原点$O(0,0,0)$为一个顶点长方体的内部点的数量,像这样的长方体可以用一个坐标$(x,y,z)$表示
~\\
\par \noindent 假设当前有一个点在$t_0$时刻插入位置为$(x_0,y_0,z_0)$,如果这个点在$t$时刻一个以原点为一个端点的长方体$(x,y,z)$内部条件:
$$
t_0<t,x_0\leq x,y_0\leq y,z_0\leq z
$$
\par \noindent 由上面条件不难看出是一个4维偏序问题。
~\\
\par \noindent cdq分治套cdq分治+树状数组解决。
\begin{minted}{c++}
#include<cstring>
#include<iostream>
#include<algorithm>
using namespace std;
constexpr int N=50010;
struct Node
{
    int op;
    int x,y,z;
    int sign,id;
    int part;
}q[8*N];
int n,nn,cnt;
int b[2*N],c[N],ans[N];
int bit[2*N];
int lowbit(int x){return x&-x;}
void update(int k,int x){for(;k<=nn;k+=lowbit(k)) bit[k]+=x;}
int query(int k){int res=0;for(;k;k-=lowbit(k)) res+=bit[k];return res;}

void cdq(int l,int r)
{
    if(l>=r) return;
    int mid=l+r>>1;
    cdq(l,mid),cdq(mid+1,r);
    int i=l;
    for(int j=mid+1;j<=r;j++)
    {
        while(i<=mid&&q[i].y<=q[j].y)
        {
            if(q[i].op==0&&q[i].part==0) update(q[i].z,1);
            i++;
        }
        if(q[j].op==1&&q[j].part==1) ans[q[j].id]+=q[j].sign*query(q[j].z);
    }
    while(i>l) 
    {
        i--;
        if(q[i].op==0&&q[i].part==0) update(q[i].z,-1);
    }
    inplace_merge(q+l,q+mid+1,q+r+1,[](const Node&a,const Node&b){return a.y<b.y;});
    
}
void solve(int l,int r)
{
    if(l>=r) return;
    int mid=l+r>>1;
    solve(l,mid),solve(mid+1,r);
    
    for(int i=l;i<=mid;i++) q[i].part=0;
    for(int i=mid+1;i<=r;i++) q[i].part=1;
    stable_sort(q+l,q+r+1,[](const Node&a,const Node&b){return a.x<b.x;});
    cdq(l,r);
}
int main()
{
    ios::sync_with_stdio(false);cin.tie(nullptr);cout.tie(nullptr);
    int T; cin>>T;
    while(T--)
    {
        cin>>n;
        cnt=nn=0;
        for(int i=1;i<=n;i++) ans[i]=0,c[i]=0;
        for(int i=1;i<=n;i++)
        {
            int op;
            cin>>op;
            if(op==1)
            {
                int x,y,z;
                cin>>x>>y>>z;
                q[++cnt]={0,x,y,z};
                b[++nn]=z;
            }
            else
            {
                c[i]=1;
                int x1,y1,z1,x2,y2,z2;
                cin>>x1>>y1>>z1>>x2>>y2>>z2;
                q[++cnt]={1,x2,y2,z2,1,i};
                q[++cnt]={1,x1-1,y2,z2,-1,i};
                q[++cnt]={1,x2,y1-1,z2,-1,i};
                q[++cnt]={1,x2,y2,z1-1,-1,i};
                q[++cnt]={1,x1-1,y1-1,z2,1,i};
                q[++cnt]={1,x1-1,y2,z1-1,1,i};
                q[++cnt]={1,x2,y1-1,z1-1,1,i};
                q[++cnt]={1,x1-1,y1-1,z1-1,-1,i};
                b[++nn]=z1-1;
                b[++nn]=z2;
            }
        }
        sort(b+1,b+1+nn);
        nn=unique(b+1,b+1+nn)-b-1;
        for(int i=1;i<=cnt;i++) q[i].z=lower_bound(b+1,b+1+nn,q[i].z)-b;
        solve(1,cnt);
        for(int i=1;i<=n;i++)
            if(c[i]) cout<<ans[i]<<'\n';
    }
    return 0;
}

\end{minted}

\subsection{线段树分治}

\par \noindent 在做CDQ的时候,将询问和操作通通视为元素,在归并过程中统计左边的操作对右边的询问的贡献。
~\\
\par \noindent 而在线段树分治中,询问被固定了。按\textbf{时间轴}确定好询问的序列以后,我们还需要所有的\textbf{操作都会影响一个时间区间}。而这个区间,毫无疑问正好对应着询问的一段区间。
~\\
\par \noindent 于是,我们可以将每一个操作丢到若干询问里做区间修改了,而线段树可以高效地维护。我们开一个叶子节点下标为询问排列的线段树,作为分治过程的底层结构。
\begin{tcolorbox}
\par 给定一个 $n$ 个结点的树,每条边有一个颜色,记 $f(x,y)$表示结点 $x$ 到 $y$ 的路径上只出现了一次的颜色的数量,求 $\sum\limits_{x < y}f(x, y)$ 。数据保证 $n \le 5 \times 10^5$ 。

\par \noindent 由于我们是求所有 $f(x,y)$的总和,根据经验我们将问题转化为对每个颜色 $w$ 计算有多少路径经过恰好一条颜色为 $w$ 的边。这样每种颜色是独立的,可以分开计算。具体的:在考虑 $w$ 颜色时断开所有 颜色是 $w$ 的边,然后统计一下每一条颜色是 $w$ 边所连接两个连通块的贡献。

\end{tcolorbox}
\begin{minted}{c++}
#include<bits/stdc++.h>

using namespace std;
using ll = long long;

const int N  = 500005;

int n, fa[N], sz[N];
inline int find(int x){while(x != fa[x])x = fa[x]; return x;}
vector<pair<int,int>>t[N << 2], c[N];
ll ans;
void ins(int u, int l, int r, int L, int R, pair<int,int> w) {
    if (l >= L && r <= R)
        return t[u].push_back(w), void(); // 一些修改
    int mid = l + r >> 1;
    if (L <= mid) ins(u << 1, l, mid, L, R, w);
    if (R > mid) ins(u << 1 | 1, mid + 1, r, L, R, w);

}
void solve(int o, int l, int r) {
    vector<int> dl; // 撤销操作
    for (auto v : t[o])
    {
        int p = find(v.first), q = find(v.second);
        if (p == q) continue;
        if (sz[p] > sz[q])
            swap(p, q);
        dl.push_back(p), sz[q] += sz[p], fa[p] = q;
    }
    if (l == r)
        for (auto v : c[l]) // 考虑所颜色是 l 的边的连接的两个连通块
            ans += 1ll * sz[find(v.first)] * sz[find(v.second)];
    else
    {
        int mid = (l + r) >> 1;
        solve(o << 1, l, mid), solve(o << 1 | 1, mid + 1, r);
    }
    // 撤销 此处可以用全局的stack代替 省空间防止爆战
    reverse(dl.begin(), dl.end());
    for (auto v : dl)
        sz[fa[v]] -= sz[v], fa[v] = v;
}
int main() {
    cin >> n;
    for (int i = 1; i < n; i++) {
        int x, y, z; cin >> x >> y >>z;
        c[z].push_back({x, y});
        // 在考虑 z 颜色时断开所有 颜色是 z 的边
        if(z > 1) ins(1, 1, n, 1, z - 1, {x, y});
        if(z < n) ins(1, 1, n, z + 1, n, {x, y});
    }
    for (int i = 1; i <= n; i++) fa[i] = i, sz[i] = 1;
    solve(1, 1, n);
    
    cout << ans << '\n';
    return 0;
}
\end{minted}

\begin{tcolorbox}
\par \noindent 维护一个物品的集合,物品有重量和价值,刚开始有 $n$ 个物品,有 $q$ 个询问:
\begin{enumerate}
\item 揷入一个新物品

\item 删除一个物品

\item 询问 $\sum_{1 \leq x \leq k} s(x) a^{x-1} \bmod b$ ,其中 $s(x)$ 为重量不超过 $x$ 的物品的最大价值, $a=10^7+$ $19, b=10^9+7, k$ 由询问给出。
\end{enumerate}
\par \noindent 其中,$n \leq 5 \times 10^3, 1 \leq k \leq 10^3, 1 \leq v \leq 10^6, q \leq 3 \times 10^4$ ,至多揷入 $C=10^4$ 件物品。
\end{tcolorbox}
\par \noindent 动态维护集合的背包,支持插入和删除。物品在时间区间上有贡献,所以用线段树分治来避免合并,将某时刻的贡献和转化为分治结构路径上元素的贡献和。
~\\
\par \noindent 每个贡献被切分成 $\log$ 个结点,采用类似标记永久化的思想把贡献元素都存在线段树的结点上。每个结点的所有元素进行了一次插入,所有左侧结点的背包还进行了一次复制。元素共进行了 $\log$ 次插入(在每个结点)。总的复制次数不多于总的插入次数。
\begin{minted}{c++}
#include <bits/stdc++.h>

using namespace std;
const int N = 4e4 + 7;
const int V = 1e3 + 7;
const int base = 1e7 + 19;
const int mod = 1e9 + 7;

struct exhibits {
    int v, w, l, r;
} exb[N];
int n, q, k;
vector<pair<int,int>> seg[N<<2];
long long DP[V];

void ins (int u, int l, int r, int L, int R, pair<int,int> v) {  // 离线处理加/删除物品操作

    if(L <= l && r <= R) {
        seg[u].push_back(v);
        return ;
    }
    int mid = l + r >> 1;
    if(L <= mid) 
        ins(u<<1, l, mid, L, R, v);
    if(mid < R) 
        ins(u<<1|1, mid+1, r, L, R, v);
}

void solve (int u, int l, int r, long long * dp) { // 线段树分治 + dp

    for(auto it = seg[u].begin(); it != seg[u].end(); it++)
        for(int j = k; j >= it->second; j--) 
            dp[j] = max(dp[j], dp[j - it->second] + it->first);
            
    if(l == r) {
        long long ans = 0;
        for(int i = k; i; i--) ans = (ans * base + dp[i]) % mod;
        printf("%lld\n",ans);
        return ;
    }
    int mid = l + r >> 1;
    long long f[V];
    memcpy(f, dp, sizeof(f));
    solve(u<<1, l, mid, f);
    memcpy(f, dp, sizeof(f));
    solve(u<<1|1, mid+1, r, f);
}

int main ()
{
    int tim = 1;
    cin >> n >> k;
    for(int i = 1, v, w; i <= n; i++) {  
        // 原物品存在的时间是[1, m],其中m代表询问的数量(暂时未知,计作-1)
        cin >> v >> w;
        exb[i] = {v, w, tim, -1};
    }
    cin >> q;
    
    for(int i = 1, op, v, w, x; i <= q; i++) {
        cin >> op; 
        // 添加物品,物品消失时间未知
        if(op == 1) {  
            cin >> v >> w;
            exb[++n] = {v, w, tim, -1};  
        }  
        else if(op == 2) {
            cin >> x;
            exb[x].r = tim - 1;  // 删除物品,物品不复存在
        }
        else tim++;  // 询问操作
    }
    tim--;
    for(int i = 1; i <= n; i++) {
        exb[i].r = (exb[i].r == -1 ? tim : exb[i].r); // 消失时间仍为止 那么就是最终时间
        if(exb[i].l <= exb[i].r) 
            ins(1, 1, tim, exb[i].l, exb[i].r, make_pair(exb[i].v, exb[i].w));
    }
    solve(1, 1, tim, DP);
    return 0;
}
\end{minted}
\subsection{猫树分治}
\par \noindent 考虑某种奇怪的\textbf{序列}静态问题,我们并不会做。但是,如果所有询问的区间有交集,那么我们就能通过下列算法得出答案。
~\\
\par \noindent 选取所有询问都包含的某个位置,分别向左向右预处理某些东西。
~\\
\par \noindent 对于询问的回答,只需要在左端点取信息,在右端点取信息,再组合即可。这要求(答案/状态)能够合并。
~\\
\par \noindent \textbf{步骤:}

\par 考虑一堆询问区间和对应的状态区间 $[L,R]$,取状态区间的中点$mid=\lfloor\frac{L+R}{2}\rfloor$,从分别 mid向左右预处理某些信息。遍历所有询问,如果跨过 $(mid,mid+1)$ ,则合并左右端点信息来回答。如果询问在 $[L,mid]$ 中,则下放到左儿子。如果在 $[mid+1,R]$ 中,则下放到右儿子。
~\\
\begin{tcolorbox}
\par \noindent P6240 好吃的题目
\par \noindent 有一条小吃街,从左到右依次排列着 $n$个商店,从 1 开始标号。第 $i$个商店会只出售一种小吃,热量为 $h_i$,美味度为 $w_i$。现在有 $m$ 个吃货要来逛街,第 $i$ 个吃货会在 $[l_i,r_i]$ 的商店内寻找小吃,而且为了防止太胖,最多能摄入 $t_i$ 的热量。小吃吃多了会腻,所以同一个商店的小吃只能吃一次。现在每个吃货想知道自己最多能得到多少美味度。
\end{tcolorbox}

\begin{minted}{c++}
#include <bits/stdc++.h>
using namespace std;
using ll = long long;
const int N = 40010;
int n, m, q, v[N], w[N], b[200010];
ll f[N][205], ans[200010];
struct Node {
    int l, r, t, id;
};
void solve(int l, int r, vector<Node> &vec) {
    if (l == r) {
        for (Node a : vec) ans[a.id] = (v[l] <= a.t ? w[l] : 0);
        return;
    }
    vector<Node> vl, vr;
    int mid = l + r >> 1, idx = 0;
    for (int i = 0; i < vec.size(); i++) {
        if (vec[i].r <= mid) vl.push_back(vec[i]);
        else if (vec[i].l > mid) vr.push_back(vec[i]);
        else
            b[++idx] = i; // 跨过 mid 的区间
    }
    // 初始化
    for (int i = l; i <= r; i++) {
        for (int j = 0; j <= m; j++)
            f[i][j] = -1;
        if (i == mid || i == mid + 1)
            for (int j = 0; j <= m; j++)
                f[i][j] = (j >= v[i] ? w[i] : 0);
    }
    // 向左预处理背包
    for (int i = mid - 1; i >= l; i--)
        for (int j = 0; j <= m; j++) {
            f[i][j] = f[i + 1][j];
            if (j >= v[i])
                f[i][j] = max(f[i][j], f[i + 1][j - v[i]] + w[i]);
        }
    // 向右边预处理背包
    for (int i = mid + 2; i <= r; i++)
        for (int j = 0; j <= m; j++) {
            f[i][j] = f[i - 1][j];
            if (j >= v[i])
                f[i][j] = max(f[i][j], f[i - 1][j - v[i]] + w[i]);
        }
    // 合并左右背包的答案作为询问的答案
    for (int i = 1; i <= idx; i++) {
        int l = vec[b[i]].l, r = vec[b[i]].r, t = vec[b[i]].t;
        ll res = -1;
        for (int j = 0; j <= t; j++) res = max(res, f[l][j] + f[r][t - j]);
        ans[vec[b[i]].id] = res;
    }

    if (vl.size()) solve(l, mid, vl);
    if (vr.size()) solve(mid + 1, r, vr);
}
int main() {
    cin >> n >> q;

    for (int i = 1; i <= n; i++) cin >> v[i];
    for (int i = 1; i <= n; i++) cin >> w[i];

    vector<Node> vec;

    for (int i = 1; i <= q; i++) {
        int l, r, t;
        cin >> l >> r >> t;
        m = max(m, t);
        vec.push_back((Node) {l, r, t, i});
    }
    solve(1, n, vec);
    for (int i = 1; i <= q; i++) cout << ans[i] << '\n';
    return 0;
}
\end{minted}

