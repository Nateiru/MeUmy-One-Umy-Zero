\section{FastIO}

\begin{minted}{c++}
#pragma GCC optimize(2)
#pragma GCC optimize(3)
#pragma GCC optimize("Ofast")
// 整数读入 (int, ll)
template<class T> void read(T &x) {
    T a = 0, f = 1;
    char ch = getchar();
    while (ch < '0' || ch > '9')
        f = ch == '-' ? -1 : f, ch = getchar();
    while (ch >= '0' && ch <= '9')
        a = a * 10 + ch - '0', ch = getchar();
    x = a * f;
}
// 浮点数读入
inline double read() {
    double x = 0, y = 1.0;
    int f = 0;
    char ch = getchar();
    while (!isdigit(ch))
        f |= ch == '-', ch = getchar();
    while (isdigit(ch))
        x = x * 10 + (ch ^ 48), ch = getchar();
    ch = getchar();
    while (isdigit(ch))
        x += (y /= 10) * (ch ^ 48), ch = getchar();
    return f ? -x : x;
}
//=============================究极快读
#include<bits/stdc++.h>
using namespace std;
namespace nqio {
    const unsigned R = 4e5, W = 4e5;
    char *a, *b, i[R], o[W], *c = o, *d = o + W, h[40], *p = h, y;
    bool s;
    struct q {
        void r(char &x) { x = a == b && (b = (a = i) + fread(i, 1, R, stdin), a == b) ? -1 : *a++;}
        void f() { fwrite(o, 1, c - o, stdout);c = o;}
        ~q() { f();}
        void w(char x) { *c = x; if (++c == d) f(); }
        q &operator>>(char &x) { do r(x); while (x <= 32); return*this;}
        q &operator>>(char *x) { do r(*x);while (*x <= 32); while (*x > 32) r(*++x); *x = 0; return*this;}
        
        template<typename t>
        q &operator>>(t &x) {
            for (r(y), s = 0; !isdigit(y); r(y)) s |= y == 45;
            if (s) for (x = 0; isdigit(y); r(y)) x = x * 10 - (y ^ 48);
            else   for (x = 0; isdigit(y); r(y)) x = x * 10 + (y ^ 48);
            return*this;
        }
        q &operator<<(char x) { w(x); return*this; }
        q &operator<<(char *x) { while (*x) w(*x++); return*this; }
        q &operator<<(const char *x) { while (*x) w(*x++); return*this; }
        
        template<typename t>
        q &operator<<(t x) {
            if (!x) w(48);
            else if (x < 0) for (w(45); x; x /= 10) * p++ = 48 | -(x % 10);
            else for (; x; x /= 10) * p++ = 48 | x % 10;
            while (p != h) w(*--p);
            return*this;
        }
    } qio;
} using nqio::qio;

int main()
{
    __int128 a,b;
    qio >> a >> b;
    qio << a + b<< '\n';
    return 0;
}
\end{minted}
\section{\_\_int128 输出函数}
\begin{minted}{c++}
void output(__int128 x) {
    if (!x)
        return;
    if (x < 0)
        putchar('-'), x = -x;
    output(x / 10);
    putchar(x % 10 + '0');
}
\end{minted}

\section{开栈}
\par \noindent Tips: 并不是在所有的地方都能用。一定要最后写一句 exit(0); 退出程序。
\begin{minted}{c++}
//64−bit
int size = 1 << 20;     //256M
char *p = (char *)malloc(size) + size;
__asm__("movq %0, %%rsp\n" :: "r"(p));

//32−bit
int size = 1 << 20;     //256M
char *p = (char *)malloc(size) + size;
__asm__("movl %0, %%esp\n":: "r"(p));

// 内存屏障
asm volatile("" ::: "memory");
__asm__ __volatile__ ("" ::: "memory");
\end{minted}
\section{随机}
\begin{itemize}
\item 不要使⽤ rand()。

\item chrono::steady\_clock::now().time\_since\_epoch().count() 可⽤于计时。

\item 64位可以使⽤ mt19937\_64。
\end{itemize}
\begin{minted}{c++}
int main() {
    
    const int N=1000010;
    mt19937 rng(chrono::steady_clock::now().time_since_epoch().count());
    
    vector<int> permutation(N);
    
    for (int i = 0; i < N; i++) permutation[i] = i;
    
    shuffle(permutation.begin(), permutation.end(), rng);
    
    for (int i = 0; i < N; i++) permutation[i] = i;
    
    for (int i = 1; i < N; i++) swap(permutation[i], permutation[uniform_int_distribution<int>(0, i)(rng)]);
}
//============ 真实随机数
mt19937 mt(time(0));
auto rd = bind(uniform_real_distribution<double>(0, 1), mt);
auto rd2 = bind(uniform_int_distribution<int>(1, 6), mt);
\end{minted}