\section{整体二分}
\par \noindent 可以使用整体二分解决的题目需要满足以下性质:

\begin{enumerate}
\item 询问的答案具有\textbf{可二分性}
\item 修改对判定答案的贡献互相独立 ,修改之间互不影响效果
\item 修改如果对判定答案有贡献,则贡献为一确定的与判定标准无关的值
\item 贡献满足交换律,结合律,具有可加性
\item 题目允许使用\textbf{离线算法}
\end{enumerate}

\par \noindent 与CDQ分治类似,将询问和修改都看成“操作”。我们首先把所有操作\textbf{按时间顺序}存入数组中,然后开始分治,定义函数:
\begin{tcolorbox}
\par solve(vl, vr, ql, qr) = 在值域[vl,vr]上二分处理[ql,qr]这些操作
\end{tcolorbox}
\par \noindent 在每一层分治中,利用数据结构(常见的是树状数组)统计当前查询的答案和 mid = (vl+vr)/2 之间的关系。根据查询出来的答案和 mid 间的关系( \textbf{<=mid} 或者 \textbf{>mid} )将当前处理的操作序列分为 lq 和 rq 两份,并分别递归处理(注意修改和询问都要递归)。
~\\
\par \noindent 边界:当 vl==vr 时,找到答案,记录答案并返回即可。
~\\
\par \noindent 需要注意的是,在整体二分过程中,solve(vl, vr, ql, qr)只处理答案在[vl, vr]内的询问,最终答案范围不在 [vl, vr] 的询问会在其他solve函数中处理。solve函数其实就是在【值域线段树】上同步实现所有二分操作的过程。
\begin{minted}{c++}
// 带修改的区间第k大
struct Node {
    int op,x,y,k; // 将 x 修改为 y 或者询问 [x,y] 第 k 大
    int id;
}q[N],rq[N],lq[N];
// 当前的值域范围为 [vl,vr], 处理的操作的区间为 [ql,qr]
void solve(int vl, int vr, int ql, int qr) {
    if (ql > qr) return;

    if (vl == vr) {
        for (int i = ql; i <= qr; i++)
            if (q[i].op == 2)
                ans[q[i].id] = vl;
        return;
    }
    int mid = vl + vr >> 1;
    int l = 0, r = 0;

    for (int i = ql; i <= qr; i++) {
        if (q[i].op == 1) {     //修改
            if (q[i].y <= mid)
                lq[++l] = q[i];
            else
                change(q[i].x, q[i].k), rq[++r] = q[i];
        } else {                //询问
            int tmp = query(q[i].y) - query(q[i].x - 1);
            if (q[i].k <= tmp)  // 第 k 大在[mid+1, vr]区间
                rq[++r] = q[i];
            else                // 第 k 大在[vl,mid]区间
                q[i].k -= tmp, lq[++l] = q[i];
        }
    }
    // 撤销当前操作
    for (int i = ql; i <= qr; i++)
        if (q[i].op == 1 && q[i].y > mid)
            change(q[i].x, -q[i].k);

    for (int i = 1; i <= l; i++) q[ql + i - 1] = lq[i];
    for (int i = 1; i <= r; i++) q[ql + l + i - 1] = rq[i];

    solve(vl, mid, ql, ql + l - 1), solve(mid + 1, vr, ql + l, qr);
}
\end{minted}