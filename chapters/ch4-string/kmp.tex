\section{KMP 算法}
\par \noindent 字符串 $s$ 的\textbf{border}:若 $s$ 的一个子串既是它的前缀又是它的后缀,则这个子串是它的border(一般不包含本身)
~\\
\par \noindent 字符串 $s$ 的\textbf{period}:循环节。用前 $T$ 个字符向后不断复制,能得到 $s$,最后一次可以只复制一部分

\begin{itemize}
\item \textbf{引理1}:如果有一个border $k$ 长度大于 $s$ 的一半,可以得出得 $s$ 有周期 $|s|−|k|$
\item \textbf{引理2}:如果 $p,q$ 都为周期,则 $\gcd(p,q)$ 也为周期
\item \textbf{引理3}:字符串 $s$ 所有不小于 $|s|$ 一半的border构成一个等差数列
\item \textbf{引理4}:可以把字符串分成 $log|s|$ 段,每一段的border都是一个等差数列
\end{itemize}

\begin{minted}{c++}
#include<iostream>
using namespace std;
const int N=1000010;
int n,m;
char p[N],s[N];
int ne[N];
int main()
{
    cin>>n>>p+1>>m>>s+1;
    // 求ne过程看成两个相同的串匹配
    for(int i=2,j=0;i<=n;i++)
    {
        while(j&&p[i]!=p[j+1]) j=ne[j];
        if(p[i]==p[j+1]) j++;// i结尾能够匹配 1~j 那么ne[i]=j
        ne[i]=j;
    }
    // 当前需要判断是否匹配 p[j+1]?=s[i]
    for(int i=1,j=0;i<=m;i++)
    {
        while(j&&s[i]!=p[j+1]) j=ne[j];
        if(s[i]==p[j+1]) j++;
        if(j==n)
        {
            cout<<i-n<<' ';
            j=ne[j];
        }
    }
    return 0;
}
\end{minted}

\subsection{Border等差数列}
\par \noindent 一个字符串的所有border可以被我们分成log数量级个等差数列。
~\\
\par \noindent 在 KMP 匹配中,我们可以利用这个性质快速跳过一串border
~\\
\par \noindent 具体而言,在一次跳border时,如果发现border长度不小于原串的一半,则接下来的border构成等差数列,直到一半以下(引理3),可以直接跳到 $(x−\lfloor\frac{\lfloor\frac{x}{2}\rfloor}{d}\rfloor×d)$。(网上博客直接跳到了 $x\mod d+d$处,经过几道题检验也是对的,但不是很能理解)
~\\
\par \noindent 一次至少跳一半,保证 $\log$ 次以内可以跳完
~\\
\begin{tcolorbox}
\par 有 $m$ 组询问,每组询问给定 $p, q$,求 $s$ 的 $p$ 前缀和 $q$ 前缀 的 \textbf{最长公共border} 的长度。
\end{tcolorbox}
\begin{minted}{c++}
#include <bits/stdc++.h>

using namespace std;
const int N = 1000010;
int n, m;
char s[N];int ne[N];
int main() {
    cin >> s + 1; n = strlen(s + 1);
    for (int i = 2, j = 0; i <= n; i++) {
        while (j && s[i] != s[j + 1]) j = ne[j];
        if (s[i] == s[j + 1]) j++;
        ne[i] = j;
    }
    int m; cin >> m;
    while (m--) {
        int p, q;
        cin >> p >> q;
        p = ne[p], q = ne[q];
        while (p != q) {
            if (p < q) swap(p, q);

            if (ne[p] > p / 2) {
                int d = p - ne[p];
                if (p % d == q % d)
                    p = q;
                else
                    p = p % d + d;
            } else
                p = ne[p];
        }
        cout << p << '\n';
    }
    return 0;
}
\end{minted}
\subsection{最小重复字符矩阵}

\begin{tcolorbox}
\par $q$ 组询问,每次询问一个矩形的最小重复字符矩阵,行列hash,为了在KMP加速匹配过程
\end{tcolorbox}
\begin{minted}{c++}
#include <iostream>
using namespace std;
typedef unsigned long long ull;
const int N = 2010;

const ull P = 13331;
int n, q;
char s[N][N];
ull hr[N][N], hc[N][N], p[N];
int ne[N];
// 第 s[l~r][i] 和 s[l~r][j] 是否相等 
bool samc(int i, int j, int l, int r) {
    ull x = hc[i][r] - hc[i][l - 1] * p[r - l + 1];
    ull y = hc[j][r] - hc[j][l - 1] * p[r - l + 1];
    return (x == y);
}
// 第 s[i][l~r] 和 s[j][l~r] 是否相等 
bool samr(int i, int j, int l, int r) {
    ull x = hr[i][r] - hr[i][l - 1] * p[r - l + 1];
    ull y = hr[j][r] - hr[j][l - 1] * p[r - l + 1];
    return (x == y);
}
int main() {
    cin >> n >> q;
    for (int i = 1; i <= n; i++) cin >> s[i] + 1;
    p[0] = 1;
    for (int i = 1; i <= n; i++) p[i] = p[i - 1] * P;

    for (int i = 1; i <= n; i++)
        for (int j = 1; j <= n; j++)
            hr[i][j] = hr[i][j - 1] * P + s[i][j];

    for (int j = 1; j <= n; j++)
        for (int i = 1; i <= n; i++)
            hc[j][i] = hc[j][i - 1] * P + s[i][j];

    while (q--) {
        int a, b, c, d;
        cin >> a >> b >> c >> d;

        for (int i = 2, j = 0; i <= d - b + 1; i++) {
            while (j && !samc(i + b - 1, j + b, a, c)) j = ne[j];

            if (samc(i + b - 1, j + b, a, c))
                j++;
            ne[i] = j;
        }

        int ans = d - b + 1 - ne[d - b + 1];

        for (int i = 2, j = 0; i <= c - a + 1; i++) {
            while (j && !samr(i + a - 1, j + a, b, d)) j = ne[j];

            if (samr(i + a - 1, j + a, b, d))
                j++;
            ne[i] = j;
        }
        ans *= c - a + 1 - ne[c - a + 1];
        cout << ans << '\n';
    }

    return 0;
}
\end{minted}