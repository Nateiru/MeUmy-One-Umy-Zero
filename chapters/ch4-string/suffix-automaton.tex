\section{后缀自动机}
处理与子串相关问题的在线线性算法。
\paragraph{名词解释}
\begin{itemize}
\item  $len$ 表示从当前 $endpos$ 可向前延伸的长度最大值。设当前构造的 $SAM$ 已得到的子串为 $s$, 从当前字符向前数 $[0, len]$ 个字符得到的新子串 $t$,$t$ 一定只作为 $s$ 的后缀出现。
\item 如:对于母串 $abcdabcdbbcd$, 有三个子串 $d, cd, bcd$ 的 $endpos$ 集合相同,而从 $endpos$ 向前最多可以延伸 $3$ 个字符满足条件,所以 $len = 3$.
\item  $link$ 表示后缀连接。定义一个子串 $v \in endpos(v)$ ($endpos(v)$ 称为 $v$ 的 endpos 等价类),该等价类中长度最长的子串为 $w$, 则 $w$ 的“最长的且不在该 endpos 等价类中的后缀”记为 $t$,令 $link(v) = t$.
\item 如:对于母串 $abcdabcdcd$, $abcd, bcd$ 在同一个 endpos 等价类中。设 $v = bcd$, 则显然 $w = abcd$, $w$ 的最长的且不再该等价类中的后缀显然为 $cd$,则令 $link(bcd) = cd$. Tips: 后缀也有长度为 0 的情况,即空后缀。
\item  $nxt[i]$ 就是沿着字符 $'a' + i$ 走可以到达的下一个状态。
\end{itemize}

\paragraph{性质}

\begin{itemize}
\item 字符串 $s$ 的一个后缀自动机包含关于字符串 $s$ 的所有子串的信息。任意从初始状态 $t_0$ 开始的路径,如果我们将转移路径上的标号写下来,都会形成 $s$ 的一个 子串 。反之每个 $s$ 的子串对应从初始状态 $t_0$  开始的某条路径。
\item 每个状态 $s$ 代表的子串是区间 $[len_{link(s)}+1, len_s]$.
\item 一个长度为 $n$ 的字符串,它的 SAM 节点个数最多有 $2n-1$ 个,连边最多有 $3n-4$ 条。
\item 树形结构的性质:设字符串长度为 $n$,考虑 extend 操作中 $cur$ 变量的值(代表当前状态在节点池中的下标),该节点对应的状态是:执行 extend 操作时的当前字符串,得到的 $n$ 个节点对应了$n$ 个不同的终点,第 $i$ 个状态对应 $S_{1...i}$。如果我们将 SAM 看作一棵树,树根为 $0$ 号节点(初始状态),其余节点 $v$ 满足其父亲为该节点的后缀连接 $link(v)$。这棵树叫 $parent$ 树。
\item $parent$ 树中的每个节点的终点集合,等于其子树内所有终点节点对应的终点集合的并集。
\item $parent$ 树中,如果节点 $a$ 是 $b$ 的祖先,则节点 $a$ 对应的字符串是节点 $b$ 对应的字符串的后缀。
\item 构成的 $parent$ 树存在一些与树相关的性质,如 $S_{1...p}$ 和 $S_{1...q}$ 的最长公共后缀对应的是 $p, q$ 对应节点间的 LCA 的字符串。
\item 除了初始状态(节点 $0$)以外,每个状态 $i$ 对应的字串数量是 $len(i) - len(link(i))$,因此计算时可以自上而下计算。
\end{itemize}

\begin{minted}{c++}
namespace SuffixAutomaton {
    const int maxn = 200050;
    const int MAXLOG = 25;
    // 需要维护 right 集合时,加上以下动态开点线段树的代码
    struct Node {
        int val, l, r;
    } t[maxn * 40];
    int cnt;        // 权值线段树节点个数

    void pushup(int u) {
        t[u].val = t[t[u].l].val + t[t[u].r].val;
    }
    void update(int &u, int l, int r, int pos) {
        if (!u)
            u = ++cnt, t[u].l = t[u].r = t[u].val = 0;
        if (l == r) {
            t[u].val++;
            return;
        }
        int mid = (l + r) >> 1;
        if (pos <= mid)
            update(t[u].l, l, mid, pos);
        else
            update(t[u].r, mid + 1, r, pos);
        pushup(u);
    }
    int merge(int x, int y) {
        if (!x || !y)
            return x + y;
        int o = ++cnt;
        t[o].l = merge(t[x].l, t[y].l);
        t[o].r = merge(t[x].r, t[y].r);
        pushup(o);
        return o;
    }

    int query(int u, int l, int r, int L, int R) {
        if (!u) return 0;
        if (L <= l && r <= R) return t[u].val;
        int mid = l+r>>1;
        int v = 0;
        if (L < =mid) v += query(t[u].l, l, mid, L, R);
        if (R > mid) v += query(t[u].r, mid+1, r, L, R);
        return v;
    }
    /* 后缀自动机 */
    struct State {
        int len, link, ch[26];
        State(int _len = 0, int _link = 0): len(_len), link(_link) {
            memset(ch, 0, sizeof ch);
        }
    } st[maxn << 1];    // 最多有 2n-1 个节点,开两倍空间

    // tot: 状态个数,last: 上一次插入的字符对应状态,sum:当前产生子串个数,n 字符串长度
    // sa c 基数排序数组, endpos[i] 表示 i 状态所代表的 endpos 集合线段树的树根
    int last, tot;
    int n, sum;
    int endpos[maxn << 1], sa[maxn << 1], c[maxn << 1], pos[maxn << 1];
    int f[maxn << 1][MAXLOG];
    int ans[maxn << 1];
    int extend(int c, int idx) {
        int p = last;
        int np = last = ++ tot;
        
        st[np] = State(st[p].len + 1);
        ans[np] = 1;

        endpos[np] = 0;
        update(endpos[np], 1, n, idx);  // 更新当前点的 endpos,注意权值线段树值域范围

        for (; p && !st[p].ch[c]; p=st[p].link) st[p].ch[c] = np;
        
        if (!p) st[np].link = 1;
        else {
            int q = st[p].ch[c];
            if (st[q].len == st[p].len + 1)
                st[np].link = q;
            else {
                int clone = ++tot;
                st[clone] = State(st[p].len + 1, st[q].link);
                memcpy(st[clone].ch, st[q].ch, sizeof st[q].ch);
                
                endpos[clone] = 0;              // 为克隆节点新建 endpos,但不建树
                for (;p && st[p].ch[c] == q; p = st[p].link) st[p].ch[c] = clone;
                st[q].link = st[np].link = clone;
            }
        }
        sum += st[np].link ? st[np].len : st[np].len - st[st[np].link].len;   // 字串个数
        return sum;          
    }
    
    // 基于基数排序的拓扑排序,保证状态间的拓扑关系,即子状态在后,父状态在前
    // 后缀自动机更新信息时,需要先更新子状态 s,再更新父状态 link[s]
    void toposort() {
        memset(sa, 0, sizeof sa);
        memset(c, 0, sizeof c);
        for (int i = 1; i <= tot; i++)
            c[st[i].len]++;            // 排序的关键字是 len
        for (int i = 1; i <= tot; i++)
            c[i] += c[i-1];
        for (int i = 1; i <= tot; i++)
            sa[c[st[i].len]--] = i;
    }
    // 建立后缀自动机
    void build(char s[]) {
        n = strlen(s + 1);
        for (int i = 1; i <= n; i++) {
            extend(s[i] - 'a', i);
            pos[i] = last;
        }
        // 预处理倍增表
        for (int i = 1; i <= tot; i++) f[i][0] = st[i].link;
        for (int j = 1; j < MAXLOG; j++)
            for (int i = 1; i <= tot; i++)
                f[i][j] = f[f[i][j-1]][j-1];

        toposort();
        // 如果需要维护 right 集合:从子节点开始,合并 endpos 集合
        for (int i = tot; i > 1; i--) {
            int u = sa[i];
            if (st[u].link)
                endpos[st[u].link] = merge(endpos[st[u].link], endpos[u]);
        }
      // 按照 len 拓扑排序,能够递推求出记录每个节点的最左边出现的位置endpos和最右边出现的位置endpos。
    }
    // 询问子串 s[l,r] 在子串 s[L,R] 中出现的次数。
    int solve(int l, int r, int L, int R) {
        // 首先从 endpos 为 r 的节点开始,倍增找到与目标串一样长的点
        int u = pos[r], nplen = r - l + 1;
        for (int i = MAXLOG - 1; i >= 0; i--)
            if (st[f[u][i]].len >= nplen)
                u = f[u][i];
        return query(endpos[u], 1, n, L + r - l, R);
    }
    void init() {
        sum = 0, tot = 0,
        st[last = ++tot] = State(0, 0);
        memset(t, 0, sizeof t);     // 清空权值线段树
    }
} using namespace SuffixAutomaton;
\end{minted}

\subsection{应用}

\subsubsection{不同子串的数目}
\par \noindent SAM,其实就是统计所有状态包含的子串总数,也就是 $\sum_i \text{maxlen}[i]-\text{minlen}[i]+1$,而实际上知道$\text{minlen}[i]=\text{maxlen}[fa[i]]+1$。
~\\
\par \noindent 每个 S 的子串都相当于自动机中的以S0为起点的一些路径。因此不同子串的个数等于自动机中\textbf{以 S0 为起点的不同路径的条数}。考虑到 SAM 为有向无环图,不同路径的条数可以通过动态规划计算。具体的:即令 $d_u$ 为从状态 u 开始的路径数量(或称之不同子串数目),对于节点u,u如果通过字母c转移到了后继v,有方程:
$$
d_u=1+\sum_{(u,v,c)\in\text{DAWG}} d_v
$$
\par \noindent 可以通过dfs一下根节点向下记忆化搜索实现,也可以\textbf{逆拓扑序}往回更新实现。
~\\
\par \noindent 其实通过此我们有两种角度理解SAM上的点也就是状态
\begin{enumerate}
\item endpos:每个状态维护一些endpos相同的子串,维护的子串数量$\text{maxlen}[i]-\text{minlen}[i]+1$
\item 路径:从起点S0到该状态的一条路径唯一对映一个子串,并且路径数量等于维护的子串数量。
\end{enumerate}
\subsubsection{所有不同子串的总长度}

\par \noindent 类似的令$f_u$ 为从状态 u 开始的不同子串的总长度,对于 v 状态的所有子串都可以在前缀加上字符c,一共多出$d_v$,+1是只考虑字母c的这一条路径。(Oiwiki上没+1是因为 $d_u$ 包括空字串)
$$
f_u=1+\sum_{(u,v,c)\in\text{DAWG}} (d_v+f_v)
$$
\subsubsection{任意子串的数目}
\par \noindent 首先考虑一个子串出现的次数,不难发现就是它 endpos 集合的大小。所以我们当前需要计算的就是 $\forall st|endpos(st)|$ 的大小。如果我们每次构建时候维护这个的话,每次需要按着fa边跳到 s 更新所有状态的 endpos,然后\textbf{判断子串属于哪个状态}就能得出次数(回忆:每个状态 endpos 集合相同,endpos集合维护的是相同子串尾出现的位置)

\par \noindent 由于fa边构成一颗树,前面讲过了我们每次是暴力把路径上的所有点权值 +1 。
\par \noindent 我们就能转化成 DAG 每一个点对于它能走的路径上的所有点 +1 ,这个直接考虑在 DAG 图上进行拓扑 dp 就行了。
~\\
\par \noindent 但注意 $\text{clone}$ 的节点是不能对它到 S 的路径上有单独贡献的,因为它的贡献会在它的本体上计算一遍。
\subsubsection{字典序第 K 大子串}

\par \noindent 字典序第 K 大的子串对应于 SAM 中字典序第 K 大的路径,因此在计算每个状态的路径数后。我们可以很容易地从 SAM 的根开始找到第 K 大的路径。
~\\
\par \noindent 如果不同位置的相同子串算作多个,那么还需要提前求出任意子串的数目(建立后缀树,令前缀的cnt=1,dfs更新一遍),否则只需要让记每个状态的 cnt=1
\subsubsection{第一次出现的位置}

\begin{tcolorbox}
给定一个文本串 S,多组查询。每次查询字符串 P 在字符串 S 中第一次出现的位置( 的开头)。
\end{tcolorbox}
\par \noindent 我们构造一个后缀自动机。我们对 SAM 中的所有状态预处理位置 $\text{firstpos}$ 。即,对每个状态 u 我们想要找到第一次出现这个状态的末端的位置 $\text{firstpos}[u]$。换句话说,我们希望先找到每个集合 $\text{endpos}$ 中的最小的元素。
~\\
\par \noindent 当我们创建新状态 np 时,我们令:

$$
\text{firstpos}(np)=\text{len}(np)
$$
$$
\text{firstpos}(nq=\text{clone})=\text{len}(q)
$$
\subsubsection{最短的没有出现的字符串}

\begin{tcolorbox}
给定一个字符串 S 和一个特定的字符集,我们要找一个长度最短的没有在 S 中出现过的字符串。
\end{tcolorbox}

\par \noindent 在SAM上做dp,设 $\text{dp}[i]$表示到点 $i$ 时需要添加的最短长度的字符满足题意。如果这个点有不是S中字符的出边,则 $\text{dp}[i]=1$(添加一个未出现的字符)否则,不光要添加一个字符还需要向后继续寻找
$$
\text{dp}[i]=1+\min\{\text{dp}_{(i,j,c)\in\text{SAM}}[j]\}
$$

\par \noindent 答案就是$\text{dp}[s0]$(起点)

\subsubsection{两个串的最长公共子串}
\par \noindent 第一个串建SAM,然后一个一个串处理。在处理每一个串的时候记录当前节点的最大匹配长度,并且记录最大长度的最小值,就是所有串的匹配长度。处理串时能在自动机上走就走,否则跳fa,类似KMP和AC自动机。

\begin{minted}{c++}
int p=1,l=0;
for(int j=0;t[j];j++)
{
    int c=t[j]-'a';
    while ( p>1 && !sam[p].ch[c]) p = sam[p].fa,l = sam[p].len;
    if (sam[p].ch[c]) p = sam[p].ch[c], l++;
    len[p] = max(len[p], l);
}
\end{minted}
\subsubsection{求子串 $[l, r]$ 在子串 $[L, R]$ 的出现次数}
\par \noindent 实际上是求子串 $s[l,r]$ 的 endpos 集合中在 $[L+r-l,R]$ 出现的次数,也就是需要维护 endpos 集合,对于一个SAM状态 endpos 集合 应该是所有子树 endpos 集合的并集 + “本身的”(如果改该状态对于是一个前缀)。
\subsubsection{区间 $[l, r]$ 本质不同子串个数}

\par \noindent 考虑把本质相同的子串看作同一种颜色。「静态区间不同颜色种类数」的经典问题的加强版!
~\\
\par \noindent 插入位置 $i$ 时应该在加入所有以 $i$ 结尾的子串的贡献。子串是有长度的,但我们只需维护左端点即可。那么「在线段树中把当前位置 +1」可以直接一次区间修改来完成,而把「上一个相同元素的位置 −1」目前来看不太好做,因为我们还不知道每个子串上一次出现的位置。
~\\
\par \noindent 为了解决这个问题,我们对原串建立后缀自动机。
~\\
\par \noindent 以 $i$ 结尾的子串就是前缀 $i$ 对应的节点在 $\text{parent}$ 树上的所有祖先节点。由同一个状态表示的子串,它们「上一次出现的位置」的右端点是相同的,而左端点是连续的一段。
~\\
\par \noindent 可以通过暴力跳 $\text{parent}$ 树上祖先并同时区间修改(增删贡献)来达到目的。同时还需要把这条链上的节点都染成 $i$ 颜色,表示把这些子串最后一次出现的位置修改为 $i$。更具体的:
~\\
\par \noindent $\text{parent}$ 树上如果该节点之前被染成 $r$ 颜色,说明子串$[r-maxlen+1\to r,   r]$ 都计算过贡献,如果本次需要将其染成 $i$ 颜色 $i > r$,那么说明 $r, i$ 都是该状态的 $\text{endpos}$ 集合,并且子串 $[r-maxlen+1\to r,   r]$ 和 $[i-maxlen+1\to i,   i]$ 是本质相同的子串!
~\\
\par \noindent 离线下来扫描区间右端点然后询问左端点即可。
~\\
\par \noindent 注意:后缀自动机的状态维护:①相同结尾的连续子串 ② endpos 集合相同的子串
~\\
\par \noindent 发现颜色相同的节点的节点会连成一段,我们可以将它们一起处理。由于只有「将某一点到根节点的颜色染成一种没有出现过的颜色」这一种操作,所有需要处理的链上的总颜色数实际上是 $O(n\log n)$ 的。原因是染色操作其实对应着 LCT 的 $\text{Access}$ 操作,可以套用其复杂度证明方法。所以在实现时我们也可以直接使用 LCT 来维护,因为一条实链上的颜色一定都是相同的,直接模拟 $\text{Access}$ 的过程即可。

\subsection{LCT维护Parent树}
\begin{minted}{c++}
// P6292-区间本质不同子串个数 https://www.luogu.com.cn/problem/P6292

#include<bits/stdc++.h>

using namespace std;
using ll=long long;

const int N=200005;
struct SAM
{
    int ch[26],fa,len;
}sam[N];
int tot=1,last=1;
int extend(int c)
{   
    int p=last;
    int np=last=++tot;
    sam[np].len=sam[p].len+1;
    for(;p&&!sam[p].ch[c];p=sam[p].fa) sam[p].ch[c]=np;

    if(!p) sam[np].fa=1;
    else
    {
        int q=sam[p].ch[c];
        if(sam[q].len==sam[p].len+1) sam[np].fa=q;
        else
        {
            int nq=++tot;sam[nq]=sam[q];
            sam[nq].len=sam[p].len+1;
            sam[q].fa=sam[np].fa=nq;
            for (;p&&sam[p].ch[c]==q;p=sam[p].fa) sam[p].ch[c]=nq;
        }
    } 
    return last;
}

const int maxn = 200005;
namespace Fenwick{ //区间加 区间询问
    ll c0[maxn],c1[maxn];
    void add(int k,int v)
    {
        ll i=k*v;
        while(k<=maxn)
        {
            c0[k]+=v;
            c1[k]+=i;
            k+=k&-k;
        }
    }
    void add(int l,int r,int v){add(l,v);add(r+1,-v);}
    ll qry(int k)
    {
        ll v=0;
        ll k0=k+1,k1=-1;
        while(k)
        {
            v+=k0*c0[k]+k1*c1[k];
            k-=k&-k;
        }
        return v;
    }
    ll qry(int l,int r){return qry(r)-qry(l-1);}

}
//using namespace Fenwick;

//若要修改一个点的点权,应当先将其 splay 到根,然后修改,最后还要调用 pushup 维护。
namespace lct {
    int ch[maxn][2], fa[maxn], stk[maxn], rev[maxn];

    int co[maxn],tag[maxn];
    int val[maxn],len[maxn];

    void init() { //初始化 link-cut-tree
        memset(ch, 0, sizeof(ch));
        memset(fa, 0, sizeof(fa));
        memset(rev, 0, sizeof(rev));
        memset(co, 0, sizeof(co));
        memset(tag, 0, sizeof(tag));
        memset(val,0,sizeof (val));
        memset(len,0,sizeof (len));
        val[0]=0x3f3f3f3f;

    }
    inline bool son(int x) {
        return ch[fa[x]][1] == x;
    }
    inline bool isroot(int x) {
        return ch[fa[x]][1] != x && ch[fa[x]][0] != x;
    }
    inline void reverse(int x) { //给结点 x 打上反转标记
        swap(ch[x][1], ch[x][0]);
        rev[x] ^= 1; 
    }
    inline void cover(int x,int color)
    {
        co[x]=tag[x]=color;
    }
    inline void pushup(int x) { 
        val[x]=min({val[ch[x][1]],val[ch[x][0]],len[x]});
    }
    inline void pushdown(int x) {
        if (rev[x]) {
            reverse(ch[x][0]);
            reverse(ch[x][1]);
            rev[x] = 0;
        }
        if(tag[x]) {
            cover(ch[x][0],tag[x]);
            cover(ch[x][1],tag[x]);
            tag[x]=0;
        }
    }
    void rotate(int x) {
        int y = fa[x], z = fa[y], c = son(x);
        if (!isroot(y))
            ch[z][son(y)] = x;
        fa[x] = z;
        ch[y][c] = ch[x][!c];
        fa[ch[y][c]] = y;
        ch[x][!c] = y;
        fa[y] = x;
        pushup(y);
    }
    void splay(int x) { // 将x设置为spaly的根节点
        int top = 0;
        stk[++top] = x;
        for (int i = x; !isroot(i); i = fa[i])
            stk[++top] = fa[i];
        while (top)
            pushdown(stk[top--]);
        for (int y = fa[x]; !isroot(x); rotate(x), y = fa[x]) if (!isroot(y))
            son(x) ^ son(y) ? rotate(x) : rotate(y);
        pushup(x);
    }
    void access(int x) { // 建立从根到 x 的路径
        for (int y = 0; x; y = x, x = fa[x]) {
            splay(x);
            ch[x][1] = y;
            pushup(x);
            if(co[x]){
                Fenwick::add(co[x]-sam[x].len+1,co[x]-val[x]+1 ,-1);
                //Fenwick::add(co[x]-sam[x].len+1,co[x]-sam[fa[x]].len ,-1);
            }
        }
    }
}
using namespace lct;

int n,m;
char s[N];
vector<pair<int,int>> q[N];
int pos[N];
ll ans[N];
int main()
{
    cin>>s+1;
    n=strlen(s+1);
    cin>>m;
    for(int i=1;i<=m;i++)
    {
        int l,r;
        cin>>l>>r;
        q[r].push_back({l,i});
    }
    for(int i=1;i<=n;i++) pos[i]=extend(s[i]-'a'); 
    init();
    for(int i=1;i<=tot;i++) 
    {
        fa[i]=sam[i].fa;
        val[i]=len[i]=sam[sam[i].fa].len+1;
    }
    for(int r=1;r<=n;r++)
    {
        access(pos[r]);
        splay(pos[r]);
        cover(pos[r],r);
        Fenwick::add(r-sam[pos[r]].len+1,r,1);
        for(auto [l,id]:q[r])  ans[id]=Fenwick::qry(l,r);
        
    }
    for(int i=1;i<=m;i++) cout<<ans[i]<<'\n';

    return 0;
}
\end{minted}
\clearpage