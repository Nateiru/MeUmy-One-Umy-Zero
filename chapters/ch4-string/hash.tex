\section{Hash}
\subsection{字符串哈希}
\begin{minted}{c++}
#include<bits/stdc++.h>

using namespace std;
typedef unsigned long long ll;

const int N=1e5+5;

struct Hash {
    const int base = 131;           // 13331 19260817 防止哈希冲突
    const ll p = 212370440130137957;// (1ull<<61)-1 {402653189,805306457,1610612741,998244353}
 // 模两个质数的双哈希,建议取一些比较不常见的质数,比如 122420729,163227661,217636919, 1222827239, 998244353 等
    ll Pow[N], h[N];
 
    inline ll mul(ll x, ll y) {
        return (x * y - (ll) ((long double) x / p * y) * p + p) % p;
    }
    inline ll add(ll x, ll y) { return (x += y) >= p ? x - p : x; } 
 
    void init(char *s) {
        int l = strlen(s + 1); 
        Pow[0] = 1; 
        for (int i = 1; i <= l; ++i) Pow[i] = mul(Pow[i - 1], base);
        for (int i = 1; i <= l; ++i) h[i] = add(s[i], mul(h[i - 1], base));
    }
    ll get(int l, int r) { return add(h[r], p - mul(h[l - 1], Pow[r - l + 1])); }
} h;
//=======================双哈希
const basePrime1 = 233;
const basePrime2 = 13331;
// 模两个质数的双哈希,建议取一些比较不常见的质数,比如 122420729,163227661,217636919, 1222827239, 998244353 等
pair<int, int> double_hash(char *s, int n, int mod1, int mod2) {
    int hash1 = 0, hash2 = 0;
    for (int i = 0; i < n; i++)
        hash1 = (1LL * hash * basePrime1 + s[i]) % mod1,
        hash2 = (1LL * hash * basePrime2 + s[i]) % mod2;
    return make_pair(hash1, hash2);
}
\end{minted}
\subsection{手写Hash + 集合 Set(Set自带Hash)}

\par \noindent 拉链法也称开散列法(open hashing)。
~\\
\par \noindent 拉链法是在每个存放数据的地方开一个链表,如果有多个键值索引到同一个地方,只用把他们都放到那个位置的链表里就行了。查询的时候需要把对应位置的链表整个扫一遍,对其中的每个数据比较其键值与查询的键值是否一致。
\begin{tcolorbox}
\par \noindent 2020 CCPC秦皇岛 J. Jewel Splitting:https://codeforces.com/gym/102769/problem/J(哈希的各种技巧)
\end{tcolorbox}
\begin{minted}{c++}
#include<bits/stdc++.h>

using namespace std;
const int maxn = 310000;

const unsigned long long x = 23333;//131;
const long long mod = 998244353;
unsigned long long H[maxn], xp[maxn];
long long fac[maxn], inv[maxn];
char s[maxn];
int n;
void init() {
    n = strlen(s);
    H[n] = 0;
    for (int i = n - 1; i >= 0; --i)
        H[i] = H[i + 1] * x + s[i];// - 'a' + 1;
    xp[0] = 1;
    for (int i = 1; i <= n; ++i)
        xp[i] = xp[i - 1] * x;
}

inline unsigned long long Hash(int i, int d) {
    return H[i] - H[i + d] * xp[d];
}

//key_t应当为整数类型,且实际值必须非负
template<typename key_t, typename type> struct hash_table {
    static const int maxn = 610000;        // 哈希表的大小
    static const int table_size = 800;    // 索引的范围
    int first[table_size], nxt[maxn], sz; // init: memset(first, 0, sizeof(first)), sz = 0
    key_t id[maxn];
    type data[maxn];
    hash_table() {
        init();
    }
    void init(){
        memset(first, 0, sizeof(first));
        sz = 0;
    }
    type& operator[] (key_t key) {
        const int h = key % table_size;
        for (int i = first[h]; i; i = nxt[i])
            if (id[i] == key)
                return data[i];
        int pos = ++sz;
        nxt[pos] = first[h];
        first[h] = pos;
        id[pos] = key;
        return data[pos] = type();
    }
    bool count(key_t key) {
        for (int i = first[key % table_size]; i; i = nxt[i])
            if (id[i] == key)
                return true;
        return false;
    }
    type get(key_t key) { //如果key对应的值不存在,则返回type()。
        for (int i = first[key % table_size]; i; i = nxt[i])
            if (id[i] == key)
                return data[i];
        return type();
    }
};

struct Set {
    hash_table<unsigned long long, int> d;
    // 这个集合的 Hash
    unsigned long long code = 0;
    long long result = 1;// 题目需要
    void init() {
        code = 0;
        result = 1;
        d.init();
    }
    void insert(unsigned long long x) {
        auto val = d[x];
        d[x] += 1;
        result = result * fac[val] % mod * inv[val + 1] % mod; 
        unsigned long long t = val + 37;
        code += x * x * x + 7 * x * x + 3 * x + 7;
    }
    void erase(unsigned long long x) {
        auto val = d[x];
        d[x] -= 1;
        result = result * fac[val] % mod * inv[val - 1] % mod;
        unsigned long long t = val + 37;
        code -= x * x * x + 7 * x * x + 3 * x + 7;
    }
    auto hash() {
        return code;
    }
};
int main() {
    
    fac[0]= inv[1] = 1;
    for (int i = 1; i < maxn; ++i) fac[i] = fac[i - 1] * i % mod;
    for (int i = 2; i < maxn; i++) inv[i] = (long long)inv[mod % i] * (mod - mod / i) % mod;
    
    int T, kase = 0;
    scanf("%d", &T);
    Set S;
    hash_table<unsigned long long, int> vis;
    while (T--) {
        scanf("%s", s);
        init();
        long long ans = 0;
        for (int d = 1; d <= n; d++) {
            S.init();
            vis.init();
            for (int i = 0; i + d <= n; i += d) S.insert(Hash(i, d)); // hash [i, i + d)
            vis[S.hash()]++;
            
            ans += fac[n / d] * S.result % mod;
            int dis = n % d;
            if (dis == 0) continue;
            
            for (int i = n - dis - d; i >= 0; i -= d) {
                S.erase(Hash(i, d));
                S.insert(Hash(i + dis, d));
                if (!vis.count(S.hash())) {
                    vis[S.hash()]++;
                    ans += fac[n / d] * S.result % mod;
                }
            }
        }
        ans %= mod;
        printf("Case #%d: %lld\n", ++kase, ans);
    }
    return 0;
}
\end{minted}
\subsection{树Hash}
\begin{tcolorbox}
\par \noindent 给定一棵以点 1 为根的树,你需要输出这棵树中最多能选出多少个互不同构的子树。
~\\
\par \noindent 两棵有根树 T1、T2 同构当且仅当他们的大小相等,且存在一个顶点排列 σ 使得在 T1 中 $i$ 是 $j$ 的祖先当且仅当在 T2 中 $σ(i)$ 是 $σ(j)$ 的祖先。
\end{tcolorbox}
\begin{minted}{c++}
#include <bits/stdc++.h>
using namespace std;

typedef unsigned long long ull;

mt19937_64 rnd(chrono::steady_clock::now().time_since_epoch().count());
ull base = rnd();
const int N = 1e6 + 5;
int n, tot, sz[N];
ull h[N];
vector<int> e[N];

ull H(ull x) {
    return x * x * x * 11451419 + 19260817;
}
ull F(ull x) {
    return
        H(x & ((1ll << 32) - 1)) +
        H(x >> 32);
}
// 注释一种双Hash,需要预处理质数
void dfs(int u, int fa) {
    h[u] = base;
    // f1[u] = 1;
    // f2[u] = 0;
    for (int v : e[u]) {
        if (v == fa) continue;
        dfs(v, u);
        h[u] += F(h[v]);
        // f1[u] += f1[v] * prime[sz[v]];
        // f2[u] += f2[v] * base + sz[v];
    }
    // if (sz[u] == 1) f2[u] = 1;
}
int main() {
    scanf("%d", &n);
    for (int i = 1, u, v; i < n; ++i) {
        scanf("%d%d", &u, &v);
        e[u].push_back(v);
        e[v].push_back(u);
    }
    dfs(1, 0);
    sort(h + 1, h + 1 + n);
    printf("%d\n", unique(h + 1, h + 1 + n) - h - 1);
    return 0;
}
\end{minted}