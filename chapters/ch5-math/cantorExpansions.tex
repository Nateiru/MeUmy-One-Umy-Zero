\section{康托展开}
\par \noindent 康托展开可以用来求一个 $1\sim n$ 的任意排列的字典序排名。
~\\
\par \noindent 其实康托展开的原理很简单。设有排列 $p=a_1a_2…a_n$ ,那么对任意字典序比 $p$ 小的排列,一定存在 $i$ ,使得其前 $i−1 (1\leq i<n)$ 位与 $p$ 对应位相同,第 $i$ 位比 $p_i$ 小,后续位随意。于是对于任意 i ,满足条件的排列数就是从后 $n−i+1$ 位中选一个比 $a_i$ 小的数、并将剩下 $n−i$ 个数任意排列的方案数,即为 $A_i\cdot(n−i)!$ ( $A_i$ 表示 $a_i$ 后面比 $a_i$ 小的数的个数)。遍历 $i$ 即得总方案数$\sum_{i=1}^{n−1}A_i\cdot(n−i)!$ ,再加 $1$ 即为排名。
~\\
\par \noindent 其中问题转化成如何求 $A_i$,树状数组显然可以 $O(n\log n)$ 解决此问题。 
\par \noindent \rule[-10pt]{17.5cm}{0.05em}
~\\
\par \noindent 与康托展开相对应的是\textbf{逆康托展开},即求指定排名的排列。原理也很简单,注意到
$$
n!=n(n-1)!=(n-1)\cdot(n-1)! + (n-1)! = \sum_{i=1}^{n-1}i\cdot i!
$$
而$A_i\leq n-i$,所以
$$
\sum_{i=j}^{n-1}A_i\cdot (n-i)!\leq\sum_{i=j}^{n-1}(n-i)\cdot (n-i)!=\sum_{i=1}^{n-j}i\cdot i!=(n-j+1)!
$$
\par \noindent 这意味着对于这个和式而言,每一项的 $(n−i)!$ 都比后面所有项的总和还大。于是可以用类似进制转换的方法,不断地模、除,来得到 $A$ 的每一项。
~\\
\par \noindent 得到 $A$ 后,我们已经知道每一项之后有多少个比该项小的数,也就是说 $p_i$ 就是剩余未用的数中第 $A_{i}+1$ 小的。可以朴素地实现:
\begin{minted}{c++}
ll fac[maxn], P[maxn], A[maxn];        // fac需要在外部初始化
void decanter(ll x, int n) {           // x为排列的排名,n为排列的长度
    x--;
    vector<int> rest(n, 0);
    iota(rest.begin(), rest.end(), 1); // 将rest初始化为1,2,...,n
    /** for (int i = 1; i <= n; ++i) ins(i); **/
    for (int i = 1; i <= n; ++i) {
        A[i] = x / fac[n - i];
        x %= fac[n - i];
    }

    for (int i = 1; i <= n; ++i) {
        P[i] = rest[A[i]]; /**P[i] = kth(A[i] + 1);remove(P[i]);**/
        rest.erase(lower_bound(rest.begin(), rest.end(), P[i]));
    }
}
\end{minted}

\par \noindent 当然,也可以使用各种平衡树来优化到 $O(nlog⁡n) $.