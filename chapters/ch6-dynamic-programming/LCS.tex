\section{最长公共子序列 (LCS)}
\subsection{简单版本$O(n^2)$}
\par \noindent 状态转移方程:
$$
f(i, j) = max\begin{cases} f(i-1, j)\\ f(i, j-1) \\ f(i-1, j-1) + 1, A[i] = B[j] \end{cases}
$$
\par \noindent \textbf{当其中一个数组元素各不相同时,最长公共子序列问题(LCS)可以转换为最长上升子序列问题(LIS)进行求解}。
\subsection{位运算求LCS}

\par \noindent 上述转移方程有一个极其重要的性质:    
$$
\begin{cases}
f_{i,j} \ge f_{i-1,j} \\
f_{i,j} \ge f_{i,j-1} \\
|f_{i,j} - f_{i,j-1}| \le 1
\end{cases}
$$

\par \noindent 即 $f$ 的同一行内是 \textbf{单调不减} 并且 \textbf{相邻两个相差不超过一}。
~\\
\par \noindent 我们定义矩阵 $M$ 为 $f$ 数组每行分别 \textbf{差分} 的结果,即:
$$
f_{i,j} = \sum_{k=1}^j M_{i,j}
$$
\par \noindent 根据上述 $f$ 的性质,不难发现 $M$ 是个 \textbf{01矩阵}。那么可以直接 \textbf{压位}(类似 std::bitset)。
~\\
\par \noindent 然后考虑直接转移 $M_i$ 整行,最后 $\sum_{j}M_{|B|,j}$ 就是答案。这就是优化的基本思想。

\begin{tcolorbox}

\par \noindent \textbf{M 的实际意义}

\par 上面只提到 $M$ 是个差分数组,现在来考虑它的实际意义是什么,以便推出它的转移方式。

\par 考虑一个 $M_{i,j}$ 什么时候会是 1。观察原转移方程,发现 $f_{i,j-1}$ 方向必然不会使 $f_{i,j}$ 加一,唯一两个方向就是 $f_{i-1,j-1}$ 或 $f_{i-1,j}$。
\begin{itemize}
\item 如果是从 $f_{i,j-1} + 1$ 而来,那么说明这个位置 $A_j$ 发生了配对,从而答案 +1;

\item 如果是 $f_{i-1,j}$,仔细思考一下还是一样的,在下面总有一个位置会和上面一条相同。
\end{itemize}

\par 总而言之就是 $A_j$ \textbf{被计入答案} 了,但注意这不意味着 $M_i$ 中所有的 1 都对应一个被选中的 $A_j$。
~\\
\par 正确的理解是 $M_{i,j}$ 如果为 1,设 k 为当前位到第一位之间 1 的个数,那就说明当前一个 LCS 长度为 k 的方案,最后的一位为 j。事实我们也是只需要考虑当前 LCS 的最后一位,添加时答案只要保证在当前方案的最后一位之后即可。
\end{tcolorbox}

\begin{minted}{c++}
/*
 * Author : _Wallace_
 * Source : https://www.cnblogs.com/-Wallace-/p/bit-lcs.html
 * Problem : LOJ #6564. 最长公共子序列
 * Standard : GNU C++ 03
 * Optimal : -Ofast
 */
#include <algorithm>
#include <cstddef>
#include <cstdio>
#include <cstring>

typedef unsigned long long ULL;

const int N = 7e4 + 5;
int n, m, u;

struct bitset {
  ULL t[N / 64 + 5];

  bitset() {
    memset(t, 0, sizeof(t));
  }
  bitset(const bitset &rhs) {
    memcpy(t, rhs.t, sizeof(t));
  }

  bitset& set(int p) {
    t[p >> 6] |= 1llu << (p & 63);
    return *this;
  }
  bitset& shift() {
    ULL last = 0llu;
    for (int i = 0; i < u; i++) {
      ULL cur = t[i] >> 63;
      (t[i] <<= 1) |= last, last = cur;
    }
    return *this;
  }
  int count() {
    int ret = 0;
    for (int i = 0; i < u; i++)
      ret += __builtin_popcountll(t[i]);
    return ret;
  }

  bitset& operator = (const bitset &rhs) {
    memcpy(t, rhs.t, sizeof(t));
    return *this;
  }
  bitset& operator &= (const bitset &rhs) {
    for (int i = 0; i < u; i++) t[i] &= rhs.t[i];
    return *this;
  }
  bitset& operator |= (const bitset &rhs) {
    for (int i = 0; i < u; i++) t[i] |= rhs.t[i];
    return *this;
  }
  bitset& operator ^= (const bitset &rhs) {
    for (int i = 0; i < u; i++) t[i] ^= rhs.t[i];
    return *this;
  }

  friend bitset operator - (const bitset &lhs, const bitset &rhs) {
    ULL last = 0llu; bitset ret;
    for (int i = 0; i < u; i++){
      ULL cur = (lhs.t[i] < rhs.t[i] + last);
      ret.t[i] = lhs.t[i] - rhs.t[i] - last;
      last = cur;
    }
    return ret;
  }
} p[N], f, g;

signed main() {
  scanf("%d%d", &n, &m), u = n / 64 + 1;
  for (int i = 1, c; i <= n; i++)
    scanf("%d", &c), p[c].set(i);
  for (int i = 1, c; i <= m; i++) {
    scanf("%d", &c), (g = f) |= p[c];
    f.shift(), f.set(0);
    ((f = g - f) ^= g) &= g;
  }
  printf("%d\n", f.count());
  return 0;
}
\end{minted}