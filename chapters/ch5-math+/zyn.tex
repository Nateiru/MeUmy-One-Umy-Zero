\section{Meissel–Lehmer 算法}
\par \noindent 求解 $1 \sim n$ 中质数个数和 $\pi(n)$
\begin{minted}{c++}
#include<bits/stdc++.h>
typedef long long ll;
using namespace std;
int isqrt(ll n) {
    return sqrtl(n);
}
ll count_pi(const ll N) {
    if (N <= 1) return 0;
    if (N == 2) return 1;
    const int v = isqrt(N);
    int s = (v + 1) / 2;
    vector<int> smalls(s);
    for (int i = 1; i < s; ++i)
        smalls[i] = i;
    vector<int> roughs(s);
    for (int i = 0; i < s; ++i)
        roughs[i] = 2 * i + 1;
    vector<ll> larges(s);
    for (int i = 0; i < s; ++i)
        larges[i] = (N / (2 * i + 1) - 1) / 2;
    vector<bool> skip(v + 1);
    const auto divide = [](ll n, ll d) -> int { return double(n) / d; };
    const auto half = [](int n) -> int { return (n - 1) >> 1; };
    int pc = 0;
    for (int p = 3; p <= v; p += 2)
        if (!skip[p]) {
            int q = p * p;
            if (ll(q) * q > N) break;
            skip[p] = true;
            for (int i = q; i <= v; i += 2 * p)
                skip[i] = true;
            int ns = 0;
            for (int k = 0; k < s; ++k) {
                int i = roughs[k];
                if (skip[i]) continue;
                ll d = ll(i) * p;
                larges[ns] = larges[k] - (d <= v ? larges[smalls[d >> 1] - pc] : smalls[half(divide(N, d))]) + pc;
                roughs[ns++] = i;
            }
            s = ns;
            for (int i = half(v), j = ((v / p) - 1) | 1; j >= p; j -= 2) {
                int c = smalls[j >> 1] - pc;
                for (int e = (j * p) >> 1; i >= e; --i)
                    smalls[i] -= c;
            }
            ++pc;
        }
    larges[0] += ll(s + 2 * (pc - 1)) * (s - 1) / 2;
    for (int k = 1; k < s; ++k)
        larges[0] -= larges[k];
    for (int l = 1; l < s; ++l) {
        int q = roughs[l];
        ll M = N / q;
        int e = smalls[half(M / q)] - pc;
        if (e < l + 1) break;
        ll t = 0;
        for (int k = l + 1; k <= e; ++k)
            t += smalls[half(divide(M, roughs[k]))];
        larges[0] += t - ll(e - l) * (pc + l - 1);
    }
    return larges[0] + 1;
}
int main() {
    ll N;
    scanf("%lld", &N);
    printf("%lld\n", count_pi(N));
}
\end{minted}


\section{Berlekamp–Massey 算法}
\begin{tcolorbox}
\par \noindent 给出一个数列 $P$ 从 $0$ 开始的前 $n$ 项。求序列 $P$ 在$\bmod\ 998244353$ 下的最短线性递推式,并在 $\bmod~998244353$ 下输出 $P_m$。
\end{tcolorbox}


\par \noindent 第一行共两个数 $n,m$ ,表示将会给出序列 $P$ 的前 $n$ 项,要求 $P_m$。第二行 $n$ 个数,表示 $P_0,P_1,P_2,\ldots,P_{n-1}$
~\\
\par \noindent 第一行输出该最短线性递推式。第二行输出 $P_m$ 的值。
\begin{minted}{c++}
#include<bits/stdc++.h>
#define poly vector<int>
const int N = 2e4 + 5, K = 25, mod = 998244353;
using namespace std;
int a[N];
inline char gc() {
    static char buf[1 << 16], *S, *T;
    if (S == T) {
        T = (S = buf) + fread(buf, 1, 1 << 16, stdin);
        if (S == T)return EOF;
    }
    return *(S++);
}
#define getchar gc
inline int read() {
    char h = getchar();
    int y = 0;
    while (h < '0' || h>'9')h = getchar();
    while (h >= '0' && h <= '9')y = y * 10 + h - '0', h = getchar();
    return y;
}
inline int qpow(int a, int b) {
    int j = 1;
    for (; b; b >>= 1, a = 1ll * a * a % mod)if (b & 1)j = 1ll * j * a % mod;
    return j;
}
struct Poly {//人畜无害的迷你全家桶 
    int rev[N];
    vector<int>W[2][K];
    unsigned long long tmp[N];
    static const int mod = 998244353, g = 3, g2 = 332748118, I = 86583718;
    //    static const int mod=1004535809,g=3,g2=334845270,I=483363861;
    inline int qpow(int a, int b) {
        int j = 1;
        for (; b; b >>= 1, a = 1ll * a * a % mod)if (b & 1)j = 1ll * j * a % mod;
        return j;
    }
    inline void init(int n) {
        for (int t = 2, i = 1; t <= n; i++, t <<= 1) {
            W[0][i].resize((t >> 1) + 1); W[1][i].resize((t >> 1) + 1);
            W[0][i][0] = W[1][i][0] = 1;
            if (t > 2)W[0][i][1] = qpow(g, (mod - 1) / t), W[1][i][1] = qpow(g2, (mod - 1) / t);
            for (int j = 2; j < (t >> 1); j++)W[0][i][j] = 1ll * W[0][i][j - 1] * W[0][i][1] % mod, W[1][i][j] = 1ll * W[1][i][j - 1] * W[1][i][1] % mod;
        }
    }
    inline poly val(int a)/*构建常数项*/ { poly ans(1, a); return ans; }
    inline poly add(poly a, poly b) {
        int n = a.size() - 1, m = b.size() - 1;
        a.resize(max(n, m) + 1);
        for (int i = 0; i <= m; i++)a[i] = (a[i] + b[i]) % mod;
        return a;
    }
    inline void ntt(poly &a, int n, int ty) {
        for (int i = 0; i < n; tmp[i] = a[i], i++)if (i < rev[i])swap(a[i], a[rev[i]]);
        for (int l = 1, cnt = 1; l < n; l <<= 1, cnt++) {
            for (int i = 0; i < n; i += (l << 1))
                for (int w = 1, j = 0; j < l; j++, w = W[ty][cnt][j]) {
                    int y = 1ll * w * tmp[i + j + l] % mod;
                    tmp[i + j + l] = (tmp[i + j] - y + mod); tmp[i + j] += y;
                }
            if (cnt % 20 == 0)for (int i = 0; i < n; i++)tmp[i] = tmp[i] % mod;
        }
        for (int i = 0; i < n; i++)a[i] = tmp[i] % mod;
    }
    inline poly mul(poly a, poly b, int l, bool eq)//相乘 
    {
        int n = a.size() - 1, m = b.size() - 1, s = 1, res = 0;
        while (s <= n + m)s <<= 1, res++;
        for (int i = 0; i < s; i++)rev[i] = (rev[i >> 1] >> 1) | ((i & 1) << (res - 1));
        a.resize(s); if (!eq)b.resize(s);
        ntt(a, s, 0); if (!eq)ntt(b, s, 0);
        if (!eq)for (int i = 0; i < s; i++)a[i] = 1ll * a[i] * b[i] % mod;
        else for (int i = 0; i < s; i++)a[i] = 1ll * a[i] * a[i] % mod;
        ntt(a, s, 1);
        a.resize(l == -1 ? n + m + 1 : l);
        for (int inv = qpow(s, mod - 2), i = 0; i < a.size(); i++)a[i] = 1ll * a[i] * inv % mod;
        return a;
    }
}P;
poly operator+(poly a, poly b) { return P.add(a, b); }
inline poly BM(int n) {
    poly la(0), ans(0); int mi = n + 1, p = 0, ldel = 0;
    for (int i = 0; i < n; i++) {
        int del = 0;
        for (int j = 0; j < ans.size(); j++)del = (del + 1ll * ans[j] * a[i - j - 1]) % mod;
        del = (a[i] - del + mod) % mod;
        if (del != 0) {
            if (mi == n + 1) {
                mi = 0, p = i, ldel = del;
                for (int l = 0; l <= i; l++)ans.push_back(0);
            }
            else {
                poly res(0);
                for (int l = 0; l < i - p - 1; l++)res.push_back(0);
                int inv = 1ll * del * qpow(ldel, mod - 2) % mod;
                res.push_back(inv); inv = mod - inv;
                for (int l = 0; l < la.size(); l++)res.push_back(1ll * inv * la[l] % mod);
                if (mi > (int)ans.size() - i)mi = (int)ans.size() - i, la = ans, p = i, ldel = del;
                ans = ans + res;
            }
        }
    }
    return ans;
}
poly p, q;
inline int calc(int n) {
    if (n == 0)return 1ll * p[0] * qpow(q[0], mod - 2) % mod;
    poly q2(q.size());
    for (int i = 0; i < q.size(); i++)q2[i] = (i & 1 ? mod - 1ll : 1ll) * q[i] % mod;
    p = P.mul(p, q2, -1, 0); q = P.mul(q, q2, -1, 0);
    int j, i;
    for (i = 0, j = 0; i < q.size(); i += 2, j++)q[j] = q[i]; q.resize(j);
    for (i = (n & 1), j = 0; i < p.size(); i += 2, j++)p[j] = p[i]; p.resize(j);
    return calc(n >> 1);
}
signed main() {
    int n = read(), m = read();
    for (int i = 0; i < n; i++) a[i] = read();
    q = BM(n);
    for (int i = 0; i < q.size(); i++) cout << q[i] << " "; cout << "\n";
    q.resize(q.size() + 1);
    for (int i = q.size() - 1; i; i--) q[i] = mod - q[i - 1];
    q[0] = 1;
    p.resize(q.size() - 1);
    P.init((q.size() + 1) << 3);
    for (int i = 0; i < q.size() - 1; i++) p[i] = a[i];
    p = P.mul(p, q, q.size() - 1, 0);
    //    for(int i=0;i<p.size();i++)cout<<p[i]<<" ";cout<<"!!\n";
    cout << calc(m);
}
\end{minted}

\section{第一类Stirling数——行}
\par \noindent 第一类斯特林数$\begin{bmatrix}n\\ m\end{bmatrix}$表示将 $n$ 个\textbf{不同}元素构成 $m$ 个圆排列的数目。
~\\
\par \noindent 给定n,对于所有的整数$i\in[0,n]$,你要求出$\begin{bmatrix}n\\ i\end{bmatrix}$.
~\\
\par \noindent 由于答案会非常大,所以你的输出需要对167772161($2^{25}\times 5+1$,是一个质数)取模。
~\\
\par \noindent 输入一个 $n$.
~\\
\par \noindent 你需要按顺序输出$\begin{bmatrix}n\\ 0\end{bmatrix},\begin{bmatrix}n\\ 1\end{bmatrix},\begin{bmatrix}n\\ 2\end{bmatrix},\dots,\begin{bmatrix}n\\ n\end{bmatrix}$的值。

\begin{minted}{c++}
#include <bits/stdc++.h>
typedef long long LL;
const int N = 550050;
const int mod = 167772161;
LL pow_mod(LL a, LL b) {
    LL ans = 1;
    for (; b; b >>= 1, a = a * a % mod)
        if (b & 1) ans = ans * a % mod;
    return ans;
}
int L, rev[N];
LL w[N], inv[N], fac[N], ifac[N];
void Init(int n) {
    L = 1;
    while (L <= n) L <<= 1;
    for (int i = 1; i < L; ++i)
        rev[i] = (rev[i >> 1] >> 1) | ((i & 1) * L / 2);
    LL wn = pow_mod(3, (mod - 1) / L);
    w[L >> 1] = 1;
    for (int i = L >> 1; i < L; ++i) w[i + 1] = w[i] * wn % mod;
    for (int i = (L >> 1) - 1; i; --i) w[i] = w[i << 1];
}
void DFT(LL *A, int len) {
    int k = __builtin_ctz(L) - __builtin_ctz(len);
    for (int i = 1; i < len; ++i) {
        int j = rev[i] >> k;
        if (j > i) std::swap(A[i], A[j]);
    }
    for (int h = 1; h < len; h <<= 1)
        for (int i = 0; i < len; i += (h << 1))
            for (int j = 0; j < h; ++j) {
                LL t = A[i + j + h] * w[j + h] % mod;
                A[i + j + h] = A[i + j] - t;
                A[i + j] += t;
            }
    for (int i = 0; i < len; ++i) A[i] %= mod;
}
void IDFT(LL *A, int len) {
    std::reverse(A + 1, A + len);
    DFT(A, len);
    int v = mod - (mod - 1) / len;
    for (int i = 0; i < len; ++i) A[i] = A[i] * v % mod;
}
void offset(const LL *f, int n, LL c, LL *g) {
    // g(x) = f(x + c)
    // g[i] = 1/i! sum_{j=i}^n j!f[j] c^(j-i)/(j-i)!
    static LL tA[N], tB[N];
    int l = 1; while (l <= n + n) l <<= 1;
    for (int i = 0; i < n; ++i) tA[n - i - 1] = f[i] * fac[i] % mod;
    LL pc = 1;
    for (int i = 0; i < n; ++i, pc = pc * c % mod) tB[i] = pc * ifac[i] % mod;
    for (int i = n; i < l; ++i) tA[i] = tB[i] = 0;
    DFT(tA, l); DFT(tB, l);
    for (int i = 0; i < l; ++i) tA[i] = tA[i] * tB[i] % mod;
    IDFT(tA, l);
    for (int i = 0; i < n; ++i)
        g[i] = tA[n - i - 1] * ifac[i] % mod;
}
void Solve(int n, LL *f) {
    if (n == 0) return void(f[0] = 1);
    static LL tA[N], tB[N];
    int m = n / 2;
    Solve(m, f);
    int l = 1; while (l <= n) l <<= 1;
    offset(f, m + 1, m, tA);
    for (int i = 0; i <= m; ++i) tB[i] = f[i];
    for (int i = m + 1; i < l; ++i) tA[i] = tB[i] = 0;
    DFT(tA, l); DFT(tB, l);
    for (int i = 0; i < l; ++i) tA[i] = tA[i] * tB[i] % mod;
    IDFT(tA, l);
    if (n & 1)
        for (int i = 0; i <= n; ++i)
            f[i] = ((i ? tA[i - 1] : 0) + (n - 1) * tA[i]) % mod;
    else
        for (int i = 0; i <= n; ++i)
            f[i] = tA[i];
}
LL f[N];
int main() {
    int n;
    scanf("%d", &n);
    Init(n * 2);
    inv[1] = 1;
    for (int i = 2; i <= n; ++i) inv[i] = -(mod / i) * inv[mod % i] % mod;
    fac[0] = ifac[0] = 1;
    for (int i = 1; i <= n; ++i) {
        fac[i] = fac[i - 1] * i % mod;
        ifac[i] = ifac[i - 1] * inv[i] % mod;
    }
    Solve(n, f);
    for (int i = 0; i <= n; ++i)
        printf("%lld ", (f[i] + mod) % mod);
    return 0;
}
\end{minted}
\section{第一类Stirling数——列}
\par \noindent 给定 $n$ , $k$ ,对于所有的整数$i\in[0,n]$,你要求出$\begin{bmatrix}i\\ k\end{bmatrix}$。
~\\
\par \noindent 由于答案会非常大,所以你的输出需要对167772161($2^{25}\times 5+1$,是一个质数)取模。
\begin{minted}{c++}
#include <bits/stdc++.h>
using namespace std;
#define Int register int
#define mod 167772161
#define MAXN 531072
#define Gi 3
int quick_pow(int a, int b, int c) {
    int res = 1;
    while (b) {
        if (b & 1) res = 1ll * res * a % c;
        a = 1ll * a * a % c;
        b >>= 1;
    }
    return res;
}
int limit = 1, l, r[MAXN];
void NTT(int *a, int type) {
    for (Int i = 0; i < limit; ++i) if (i < r[i]) swap(a[i], a[r[i]]);
    for (Int mid = 1; mid < limit; mid <<= 1) {
        int Wn = quick_pow(Gi, (mod - 1) / (mid << 1), mod);
        if (type == -1) Wn = quick_pow(Wn, mod - 2, mod);
        for (Int R = mid << 1, j = 0; j < limit; j += R) {
            for (Int k = 0, w = 1; k < mid; ++k, w = 1ll * w * Wn % mod) {
                int x = a[j + k], y = 1ll * w * a[j + k + mid] % mod;
                a[j + k] = (x + y) % mod, a[j + k + mid] = (x + mod - y) % mod;
            }
        }
    }
    if (type == 1) return;
    int Inv = quick_pow(limit, mod - 2, mod);
    for (Int i = 0; i < limit; ++i) a[i] = 1ll * a[i] * Inv % mod;
}
int c[MAXN];
void Solve(int len, int *a, int *b) {
    if (len == 1) return b[0] = quick_pow(a[0], mod - 2, mod), void();
    Solve((len + 1) >> 1, a, b);
    limit = 1, l = 0;
    while (limit < (len << 1)) limit <<= 1, l++;
    for (Int i = 0; i < limit; ++i) r[i] = (r[i >> 1] >> 1) | ((i & 1) << (l - 1));
    for (Int i = 0; i < len; ++i) c[i] = a[i];
    for (Int i = len; i < limit; ++i) c[i] = 0;
    NTT(c, 1); NTT(b, 1);
    for (Int i = 0; i < limit; ++i) b[i] = 1ll * b[i] * (2 + mod - 1ll * c[i] * b[i] % mod) % mod;
    NTT(b, -1);
    for (Int i = len; i < limit; ++i) b[i] = 0;
}
void deravitive(int *a, int n) {
    for (Int i = 1; i <= n; ++i) a[i - 1] = 1ll * a[i] * i % mod;
    a[n] = 0;
}
void inter(int *a, int n) {
    for (Int i = n; i >= 1; --i) a[i] = 1ll * a[i - 1] * quick_pow(i, mod - 2, mod) % mod;
    a[0] = 0;
}
int b[MAXN];
void Ln(int *a, int n) {
    memset(b, 0, sizeof(b));
    Solve(n, a, b); deravitive(a, n);
    while (limit <= n) limit <<= 1, l++;
    for (Int i = 0; i < limit; ++i) r[i] = (r[i >> 1] >> 1) | ((i & 1) << (l - 1));
    NTT(a, 1), NTT(b, 1);
    for (Int i = 0; i < limit; ++i) a[i] = 1ll * a[i] * b[i] % mod;
    NTT(a, -1);
    inter(a, n);
    for (Int i = n + 1; i < limit; ++i) a[i] = 0;
}
int F0[MAXN];
void Exp(int *a, int *B, int n) {
    if (n == 1) return B[0] = 1, void();
    Exp(a, B, (n + 1) >> 1);
    for (Int i = 0; i < limit; ++i) F0[i] = B[i];
    Ln(F0, n);
    F0[0] = (a[0] + 1 + mod - F0[0]) % mod;
    for (Int i = 1; i < n; ++i) F0[i] = (a[i] + mod - F0[i]) % mod;
    NTT(F0, 1); NTT(B, 1);
    for (Int i = 0; i < limit; ++i) B[i] = 1ll * F0[i] * B[i] % mod;
    NTT(B, -1);
    for (Int i = n; i < limit; ++i) B[i] = 0;
}
int read() {
    int x = 0; char c = getchar(); int f = 1;
    while (c < '0' || c > '9') { if (c == '-') f = -f; c = getchar(); }
    while (c >= '0' && c <= '9') { x = (int)((int)(x << 3) % mod + (int)(x << 1) % mod + c - '0') % mod; c = getchar(); }
    return x * f;
}
void write(int x) {
    if (x < 0) { x = -x; putchar('-'); }
    if (x > 9) write(x / 10);
    putchar(x % 10 + '0');
}
int n, k;
int fac[MAXN], A[MAXN], B[MAXN];
signed main() {
    n = read(), k = read();
    for (Int i = 0; i < n; ++i) A[i] = quick_pow(i + 1, mod - 2, mod);
    Ln(A, n);
    for (Int i = 0; i < n; ++i) A[i] = 1ll * A[i] * k % mod;
    Exp(A, B, n); fac[0] = 1;
    for (Int i = 1; i <= max(n, k); ++i) fac[i] = 1ll * fac[i - 1] * i % mod;
    for (Int i = n; i >= k; --i) B[i] = B[i - k];
    for (Int i = 0; i < k; ++i) B[i] = 0; int Inv = quick_pow(fac[k], mod - 2, mod);
    for (Int i = 0; i <= n; ++i) write(1ll * B[i] * fac[i] % mod * Inv % mod), putchar(' ');
    putchar('\n');
    return 0;
}
\end{minted}
\section{第二类Stirling数——行}
\par \noindent 第二类斯特林数$\begin{Bmatrix} n \\m \end{Bmatrix}$表示把 n 个\textbf{不同}元素划分成 m 个\textbf{相同}的集合中(不能有空集)的方案数。
~\\
\par \noindent 给定 $n$,对于所有的整数$i\in[0,n]$,你要求出$\begin{Bmatrix} n \\i \end{Bmatrix}$。
~\\
\par \noindent 由于答案会非常大,所以你的输出需要\textbf{对167772161($2^{25}\times 5+1$,是一个质数)取模}。
\begin{minted}{c++}
#include<bits/stdc++.h>
#include<cstdlib>
using namespace std;
#define int long long
inline void read(int &x) {
    char c = getchar(); x = 0; int f = 1;
    while (c > '9' || c < '0') { if (c == '-') f = -1; c = getchar(); }
    while (c <= '9' && c >= '0') x = (x << 1) + (x << 3) + c - '0', c = getchar();
    x = x * f;
}
const int p = 167772161ll, w = 3, N = 2e6 + 10;
inline int qpow(int a, int b) {
    int k = 1ll;
    while (b) {
        if (b & 1) k = k * a % p;
        a = a * a % p;
        b = b >> 1;
    }
    return k;
}
int inv[N], n, f[N], g[N], lim, len, rev[N];
inline int upmod(int x) {
    return (x % p + p) % p;
}
inline void ntt(int *a, int f) {
    for (int i = 0; i < lim; i++)
        if (i < rev[i]) swap(a[i], a[rev[i]]);
    for (int mid = 1; mid < lim; mid <<= 1) {
        int wn = qpow(w, (((p - 1) / (mid << 1) * f) + p - 1));
        for (int j = 0; j < lim; j += (mid << 1)) {
            int g = 1;
            for (int k = 0; k < mid; k++, g = g * wn % p) {
                int x = a[k + j], y = g * a[k + j + mid] % p;
                a[k + j] = upmod(x + y);
                a[k + j + mid] = upmod(x - y + p);
            }
        }
    }
    if (f == -1) {
        int Inv = qpow(lim, (p - 2));
        for (int i = 0; i < lim; i++) a[i] = a[i] * Inv % p;
    }
}
signed main() {
    read(n); n++;
    inv[0] = 1;
    for (int i = 1; i < n; i++) inv[i] = inv[i - 1] * i % p;
    for (int i = 1; i < n; i++) inv[i] = qpow(inv[i], p - 2);
    for (int i = 0; i < n; i++) {
        f[i] = (i & 1 ? (p - inv[i]) : inv[i]);
        g[i] = qpow(i, n - 1) * inv[i] % p;
    }
    lim = 1, len = 0;
    while (lim <= (n << 1)) len++, lim <<= 1;
    for (int i = 0; i < lim; i++) rev[i] = (rev[i >> 1] >> 1) | ((i & 1) << (len - 1));
    ntt(f, 1); ntt(g, 1);
    for (int i = 0; i < lim; i++) f[i] = f[i] * g[i] % p;
    ntt(f, -1);
    for (int i = 0; i < n; i++) printf("%lld ", f[i]);
}
\end{minted}
\section{第二类Stirling数——列}
\par \noindent 第二类斯特林数$\begin{Bmatrix} n \\m \end{Bmatrix}$表示把 $n$ 个\textbf{不同}元素划分成 $m$ 个\textbf{相同}的集合(不能有空集)的方案数。
~\\
\par \noindent 给定$n$, $k$,对于所有的整数$i\in[0,n]$,你要求出$\begin{Bmatrix} i \\k \end{Bmatrix}$。
~\\
\par \noindent 由于答案会非常大,所以你的输出需要对167772161($2^{25}\times 5+1$,是一个质数)取模。
\begin{minted}{c++}
#include<algorithm>
#include<cstdio>
#define mod 167772161
#define G 3
#define Maxn 270000
using namespace std;
int n, k, r[Maxn << 2];
long long invn, invG;
long long fac[Maxn], inv[Maxn];
long long powM(long long a, long long t = mod - 2) {
    long long ans = 1, buf = a;
    while (t) {
        if (t & 1)ans = (ans * buf) % mod;
        buf = (buf * buf) % mod;
        t >>= 1;
    }return ans;
}
void NTT(long long *f, bool op, int n) {
    for (int i = 0; i < n; i++)
        if (r[i] < i)swap(f[r[i]], f[i]);
    for (int len = 1; len < n; len <<= 1) {
        int w = powM(op == 1 ? G : invG, (mod - 1) / len / 2);
        for (int p = 0; p < n; p += len + len) {
            long long buf = 1;
            for (int i = p; i < p + len; i++) {
                int sav = f[i + len] * buf % mod;
                f[i + len] = f[i] - sav;
                if (f[i + len] < 0)f[i + len] += mod;
                f[i] = f[i] + sav;
                if (f[i] >= mod)f[i] -= mod;
                buf = buf * w % mod;
            }//F(x)=FL(x^2)+x*FR(x^2)
             //F(W^k)=FL(w^k)+W^k*FR(w^k)
             //F(W^{k+n/2})=FL(w^k)-W^k*FR(w^k)
        }
    }
}
long long g[Maxn << 2];
void rev(long long *f, int len) {
    for (int i = 0; i < len; i++)g[i] = f[i];
    for (int i = 0; i < len; i++)f[len - i - 1] = g[i];
}
//令f=f*g (mod x^lim) 
void times(long long *f, long long *gg, int len, int lim) {
    int m = len + len, n;
    for (int i = 0; i < len; i++)g[i] = gg[i];
    for (n = 1; n < m; n <<= 1); invn = powM(n);
    for (int i = len; i < n; i++)g[i] = 0;
    for (int i = 0; i < n; i++)
        r[i] = (r[i >> 1] >> 1) | ((i & 1) ? n >> 1 : 0);
    NTT(f, 1, n); NTT(g, 1, n);
    for (int i = 0; i < n; ++i)f[i] = (f[i] * g[i]) % mod;
    NTT(f, 0, n);
    for (int i = 0; i < lim; ++i)f[i] = (f[i] * invn) % mod;
    for (int i = lim; i < n; ++i)f[i] = 0;
}
void Init(int lim) {
    inv[1] = inv[0] = fac[0] = 1;
    for (int i = 1; i <= lim; i++)fac[i] = fac[i - 1] * i % mod;
    for (int i = 2; i <= lim; i++)
        inv[i] = inv[mod % i] * (mod - mod / i) % mod;
    for (int i = 2; i <= lim; i++)inv[i] = inv[i - 1] * inv[i] % mod;
    for (int i = 1; i <= lim; i++)inv[i] = powM(fac[i]);
}
long long p[Maxn << 2];
//求出F(x-c) 
void fminus(long long *s, long long *f, int len, int c) {
    c = mod - c;
    for (int i = 0; i < len; i++)
        p[len - i - 1] = f[i] * fac[i] % mod;
    long long buf = 1;
    for (int i = 0; i < len; i++, buf = buf * c % mod)
        s[i] = buf * inv[i] % mod;
    times(p, s, len, len);
    for (int i = 0; i < len; i++)s[len - i - 1] = p[i] * inv[len - i - 1] % mod;
    for (int i = len; i < len + len; i++)s[i] = 0;
}
long long f[Maxn << 2], s[Maxn << 2];
void solve(long long *f, int n) {
    if (n == 1) { f[0] = 0; f[1] = 1; }
    else if (n & 1) {
        solve(f, n - 1); f[n] = 0;
        //再乘上(x-n+1)就好了
        for (int i = n; i > 0; i--)
            f[i] = (f[i - 1] + (mod - n + 1) * f[i]) % mod;
        f[0] = f[0] * (mod - n + 1) % mod;
    }
    else {
        solve(f, n / 2);
        //S(x)=F(x+n/2)
        fminus(s, f, n / 2 + 1, n / 2);
        times(f, s, n / 2 + 1, n + 1);
    }
}
void invp(long long *f, int len) {
    for (int i = 0; i < k + 1; i++)s[i] = p[i] = 0;
    //注意清空 
    long long *r = s, *rr = p;
    int n = 1; for (; n < len; n <<= 1);
    rr[0] = powM(f[0]);
    for (int len = 2; len <= n; len <<= 1) {
        for (int i = 0; i < len; i++)
            r[i] = rr[i] * 2 % mod;
        times(rr, rr, len / 2, len);
        times(rr, f, len, len);
        for (int i = 0; i < len; i++)
            rr[i] = (r[i] - rr[i] + mod) % mod;
    }for (int i = 0; i < len; i++)
        f[i] = rr[i];
}
int main() {
    scanf("%d%d", &n, &k);
    if (k > n) {
        for (int i = 0; i <= n; i++)printf("0 ");
        return 0;
    }invG = powM(G);
    Init(k); solve(f, k + 1);
    for (int i = 0; i < k + 1; i++)f[i] = f[i + 1];
    rev(f, k + 1);
    for (int i = n - k + 1; i < k + 1; i++)f[i] = 0;
    for (int i = k + 1; i < n - k + 1; i++)f[i] = 0;
    invp(f, n - k + 1);
    for (int i = 0; i < k; i++)printf("0 ");
    for (int i = 0; i < n - k + 1; i++)printf("%lld ", f[i]);
    return 0;
}

\end{minted}
